\documentclass[]{article}
\usepackage{lmodern}
\usepackage{amssymb,amsmath}
\usepackage{ifxetex,ifluatex}
\usepackage{fixltx2e} % provides \textsubscript
\ifnum 0\ifxetex 1\fi\ifluatex 1\fi=0 % if pdftex
  \usepackage[T1]{fontenc}
  \usepackage[utf8]{inputenc}
\else % if luatex or xelatex
  \ifxetex
    \usepackage{mathspec}
  \else
    \usepackage{fontspec}
  \fi
  \defaultfontfeatures{Ligatures=TeX,Scale=MatchLowercase}
\fi
% use upquote if available, for straight quotes in verbatim environments
\IfFileExists{upquote.sty}{\usepackage{upquote}}{}
% use microtype if available
\IfFileExists{microtype.sty}{%
\usepackage{microtype}
\UseMicrotypeSet[protrusion]{basicmath} % disable protrusion for tt fonts
}{}
\usepackage[margin=1in]{geometry}
\usepackage{hyperref}
\hypersetup{unicode=true,
            pdftitle={Lab 5: The Binomial and Poisson Distributions},
            pdfborder={0 0 0},
            breaklinks=true}
\urlstyle{same}  % don't use monospace font for urls
\usepackage{color}
\usepackage{fancyvrb}
\newcommand{\VerbBar}{|}
\newcommand{\VERB}{\Verb[commandchars=\\\{\}]}
\DefineVerbatimEnvironment{Highlighting}{Verbatim}{commandchars=\\\{\}}
% Add ',fontsize=\small' for more characters per line
\usepackage{framed}
\definecolor{shadecolor}{RGB}{248,248,248}
\newenvironment{Shaded}{\begin{snugshade}}{\end{snugshade}}
\newcommand{\AlertTok}[1]{\textcolor[rgb]{0.94,0.16,0.16}{#1}}
\newcommand{\AnnotationTok}[1]{\textcolor[rgb]{0.56,0.35,0.01}{\textbf{\textit{#1}}}}
\newcommand{\AttributeTok}[1]{\textcolor[rgb]{0.77,0.63,0.00}{#1}}
\newcommand{\BaseNTok}[1]{\textcolor[rgb]{0.00,0.00,0.81}{#1}}
\newcommand{\BuiltInTok}[1]{#1}
\newcommand{\CharTok}[1]{\textcolor[rgb]{0.31,0.60,0.02}{#1}}
\newcommand{\CommentTok}[1]{\textcolor[rgb]{0.56,0.35,0.01}{\textit{#1}}}
\newcommand{\CommentVarTok}[1]{\textcolor[rgb]{0.56,0.35,0.01}{\textbf{\textit{#1}}}}
\newcommand{\ConstantTok}[1]{\textcolor[rgb]{0.00,0.00,0.00}{#1}}
\newcommand{\ControlFlowTok}[1]{\textcolor[rgb]{0.13,0.29,0.53}{\textbf{#1}}}
\newcommand{\DataTypeTok}[1]{\textcolor[rgb]{0.13,0.29,0.53}{#1}}
\newcommand{\DecValTok}[1]{\textcolor[rgb]{0.00,0.00,0.81}{#1}}
\newcommand{\DocumentationTok}[1]{\textcolor[rgb]{0.56,0.35,0.01}{\textbf{\textit{#1}}}}
\newcommand{\ErrorTok}[1]{\textcolor[rgb]{0.64,0.00,0.00}{\textbf{#1}}}
\newcommand{\ExtensionTok}[1]{#1}
\newcommand{\FloatTok}[1]{\textcolor[rgb]{0.00,0.00,0.81}{#1}}
\newcommand{\FunctionTok}[1]{\textcolor[rgb]{0.00,0.00,0.00}{#1}}
\newcommand{\ImportTok}[1]{#1}
\newcommand{\InformationTok}[1]{\textcolor[rgb]{0.56,0.35,0.01}{\textbf{\textit{#1}}}}
\newcommand{\KeywordTok}[1]{\textcolor[rgb]{0.13,0.29,0.53}{\textbf{#1}}}
\newcommand{\NormalTok}[1]{#1}
\newcommand{\OperatorTok}[1]{\textcolor[rgb]{0.81,0.36,0.00}{\textbf{#1}}}
\newcommand{\OtherTok}[1]{\textcolor[rgb]{0.56,0.35,0.01}{#1}}
\newcommand{\PreprocessorTok}[1]{\textcolor[rgb]{0.56,0.35,0.01}{\textit{#1}}}
\newcommand{\RegionMarkerTok}[1]{#1}
\newcommand{\SpecialCharTok}[1]{\textcolor[rgb]{0.00,0.00,0.00}{#1}}
\newcommand{\SpecialStringTok}[1]{\textcolor[rgb]{0.31,0.60,0.02}{#1}}
\newcommand{\StringTok}[1]{\textcolor[rgb]{0.31,0.60,0.02}{#1}}
\newcommand{\VariableTok}[1]{\textcolor[rgb]{0.00,0.00,0.00}{#1}}
\newcommand{\VerbatimStringTok}[1]{\textcolor[rgb]{0.31,0.60,0.02}{#1}}
\newcommand{\WarningTok}[1]{\textcolor[rgb]{0.56,0.35,0.01}{\textbf{\textit{#1}}}}
\usepackage{graphicx,grffile}
\makeatletter
\def\maxwidth{\ifdim\Gin@nat@width>\linewidth\linewidth\else\Gin@nat@width\fi}
\def\maxheight{\ifdim\Gin@nat@height>\textheight\textheight\else\Gin@nat@height\fi}
\makeatother
% Scale images if necessary, so that they will not overflow the page
% margins by default, and it is still possible to overwrite the defaults
% using explicit options in \includegraphics[width, height, ...]{}
\setkeys{Gin}{width=\maxwidth,height=\maxheight,keepaspectratio}
\IfFileExists{parskip.sty}{%
\usepackage{parskip}
}{% else
\setlength{\parindent}{0pt}
\setlength{\parskip}{6pt plus 2pt minus 1pt}
}
\setlength{\emergencystretch}{3em}  % prevent overfull lines
\providecommand{\tightlist}{%
  \setlength{\itemsep}{0pt}\setlength{\parskip}{0pt}}
\setcounter{secnumdepth}{0}
% Redefines (sub)paragraphs to behave more like sections
\ifx\paragraph\undefined\else
\let\oldparagraph\paragraph
\renewcommand{\paragraph}[1]{\oldparagraph{#1}\mbox{}}
\fi
\ifx\subparagraph\undefined\else
\let\oldsubparagraph\subparagraph
\renewcommand{\subparagraph}[1]{\oldsubparagraph{#1}\mbox{}}
\fi

%%% Use protect on footnotes to avoid problems with footnotes in titles
\let\rmarkdownfootnote\footnote%
\def\footnote{\protect\rmarkdownfootnote}

%%% Change title format to be more compact
\usepackage{titling}

% Create subtitle command for use in maketitle
\providecommand{\subtitle}[1]{
  \posttitle{
    \begin{center}\large#1\end{center}
    }
}

\setlength{\droptitle}{-2em}

  \title{Lab 5: The Binomial and Poisson Distributions}
    \pretitle{\vspace{\droptitle}\centering\huge}
  \posttitle{\par}
    \author{}
    \preauthor{}\postauthor{}
    \date{}
    \predate{}\postdate{}
  

\begin{document}
\maketitle

\textbf{For this lab, it will be helpful to have a copy of the knitted
version of this document to answer the questions as much of it is
written using mathematical notation that may be difficult to read when
the document is not knitted.}

\hypertarget{lab-goals}{%
\subsection{Lab Goals}\label{lab-goals}}

\begin{enumerate}
\def\labelenumi{\arabic{enumi}.}
\tightlist
\item
  The purpose of this lab is to explore the following distributions:
\end{enumerate}

\begin{itemize}
\item
  Binomial distribution
\item
  Poisson distribution
\end{itemize}

\begin{enumerate}
\def\labelenumi{\arabic{enumi}.}
\setcounter{enumi}{1}
\tightlist
\item
  You will be asked to compute probabilities of events using the pmf of
  these distributions both
\end{enumerate}

\begin{itemize}
\item
  By hand using the formula
\item
  Using R functions
\end{itemize}

Emphasis is on identifying random events that can be modeled using these
distributions and learning how to calculate probabilities using these
distributions by hand and in R.

\hypertarget{some-notation}{%
\subsubsection{Some Notation}\label{some-notation}}

Your solutions to the problems below must include the formula used for
each calculation. Here is some helpful notation you can copy, edit, and
paste as needed. This notation is \textbf{not} R code. The notation
between dollar signs in the Rmd file is in something called
\texttt{latex} which lets you write mathematical expressions nicely.
Without latex you'd get, for example, (x\^{}2+y)/z, whereas with latex
you get \(\frac{x^2+y}{z}\). Remember that code chunks are for R code
only, not \texttt{latex}.

\begin{enumerate}
\def\labelenumi{\arabic{enumi}.}
\item
  For a binomial random variable, \(X\), the pmf is given by
  \(P(X = k) = \frac{n!}{k!(n-k)!}p^{k}(1-p)^{n-k}\) for
  \(k=0,1,\ldots,n\).
\item
  For a Poisson random variable \(X\), the pmf is given by
  \(P(X = k) = e^{-\lambda}\frac{\lambda^{k}}{k!}\) for
  \(k=0,1,2,\ldots\).
\end{enumerate}

\hypertarget{some-r-code}{%
\subsection{Some R Code}\label{some-r-code}}

\begin{enumerate}
\def\labelenumi{\arabic{enumi}.}
\tightlist
\item
  \texttt{exp()} is the exponential function in R, so for instance,
  \(e^3\) (\(e\approx2.718\) to the 3rd power) is
\end{enumerate}

\begin{Shaded}
\begin{Highlighting}[]
\KeywordTok{exp}\NormalTok{(}\DecValTok{3}\NormalTok{)}
\end{Highlighting}
\end{Shaded}

\begin{verbatim}
## [1] 20.08554
\end{verbatim}

\begin{enumerate}
\def\labelenumi{\arabic{enumi}.}
\setcounter{enumi}{1}
\tightlist
\item
  \texttt{factorial()} is the R function for computing factorials, so
  for instance, \(3!=3\times2\times1\) is
\end{enumerate}

\begin{Shaded}
\begin{Highlighting}[]
\KeywordTok{factorial}\NormalTok{(}\DecValTok{3}\NormalTok{)}
\end{Highlighting}
\end{Shaded}

\begin{verbatim}
## [1] 6
\end{verbatim}

\begin{enumerate}
\def\labelenumi{\arabic{enumi}.}
\setcounter{enumi}{2}
\tightlist
\item
  In R the following operators can be used
\end{enumerate}

\begin{Shaded}
\begin{Highlighting}[]
\DecValTok{2}\OperatorTok{+}\DecValTok{2}  \CommentTok{# + for addition}
\end{Highlighting}
\end{Shaded}

\begin{verbatim}
## [1] 4
\end{verbatim}

\begin{Shaded}
\begin{Highlighting}[]
\DecValTok{3-2}  \CommentTok{# - for subtraction}
\end{Highlighting}
\end{Shaded}

\begin{verbatim}
## [1] 1
\end{verbatim}

\begin{Shaded}
\begin{Highlighting}[]
\DecValTok{3}\OperatorTok{*}\DecValTok{2}  \CommentTok{# * for multiplication}
\end{Highlighting}
\end{Shaded}

\begin{verbatim}
## [1] 6
\end{verbatim}

\begin{Shaded}
\begin{Highlighting}[]
\DecValTok{4}\OperatorTok{/}\DecValTok{4}  \CommentTok{# / for division}
\end{Highlighting}
\end{Shaded}

\begin{verbatim}
## [1] 1
\end{verbatim}

\begin{Shaded}
\begin{Highlighting}[]
\DecValTok{2}\OperatorTok{^}\DecValTok{3}  \CommentTok{# ^ for exponents}
\end{Highlighting}
\end{Shaded}

\begin{verbatim}
## [1] 8
\end{verbatim}

\begin{enumerate}
\def\labelenumi{\arabic{enumi}.}
\setcounter{enumi}{3}
\tightlist
\item
  Keep in mind that R follows the order of operations. So, when
  evaluating 3 + 6*8. Multiplication will be performed before addition.
\end{enumerate}

\begin{Shaded}
\begin{Highlighting}[]
\DecValTok{3}\OperatorTok{+}\DecValTok{6}\OperatorTok{*}\DecValTok{8}
\end{Highlighting}
\end{Shaded}

\begin{verbatim}
## [1] 51
\end{verbatim}

To have the addition evaluated first, you must use parenthesis.

\begin{Shaded}
\begin{Highlighting}[]
\NormalTok{(}\DecValTok{3}\OperatorTok{+}\DecValTok{6}\NormalTok{)}\OperatorTok{*}\DecValTok{8}
\end{Highlighting}
\end{Shaded}

\begin{verbatim}
## [1] 72
\end{verbatim}

R will evaluate an expression in the following order:

\begin{enumerate}
\def\labelenumi{\arabic{enumi})}
\item
  Parenthesis
\item
  Exponents
\item
  Multiplication
\item
  Division
\item
  Addition
\item
  Subtraction
\end{enumerate}

\begin{enumerate}
\def\labelenumi{\arabic{enumi}.}
\setcounter{enumi}{4}
\tightlist
\item
  Here is some code you can copy, paste, and edit in code chunks to
  calculate probabilities in this document for the binomial and Poisson
  distributions. DO NOT CHANGE THESE CODE CHUNKS TO \texttt{eval=TRUE}.
  Variables in them have not been defined, and it will not run.
\end{enumerate}

\begin{enumerate}
\def\labelenumi{\alph{enumi})}
\tightlist
\item
  To calculate \(P(X = k)\) for a binomial distribution use the
  following code where you have specified \texttt{n},\texttt{k}, and
  \texttt{p}.
\end{enumerate}

\begin{Shaded}
\begin{Highlighting}[]
\NormalTok{P_k =}\StringTok{ }\KeywordTok{choose}\NormalTok{(n,k)}\OperatorTok{*}\NormalTok{(p}\OperatorTok{^}\NormalTok{k)}\OperatorTok{*}\NormalTok{(}\DecValTok{1}\OperatorTok{-}\NormalTok{p)}\OperatorTok{^}\NormalTok{(n}\OperatorTok{-}\NormalTok{k)}
\end{Highlighting}
\end{Shaded}

\begin{enumerate}
\def\labelenumi{\alph{enumi})}
\setcounter{enumi}{1}
\tightlist
\item
  To calculate P(X = k) for the Poisson distribution use the following
  code where you have defined \texttt{lambda} and \texttt{k}.
\end{enumerate}

\begin{Shaded}
\begin{Highlighting}[]
\NormalTok{P_k =}\StringTok{ }\NormalTok{(}\KeywordTok{exp}\NormalTok{(}\OperatorTok{-}\NormalTok{lambda)}\OperatorTok{*}\NormalTok{lambda}\OperatorTok{^}\NormalTok{(k))}\OperatorTok{/}\KeywordTok{factorial}\NormalTok{(k)}
\end{Highlighting}
\end{Shaded}

\hypertarget{properties-of-random-variables}{%
\subsection{Properties of Random
Variables}\label{properties-of-random-variables}}

For any random variable X,

\begin{enumerate}
\def\labelenumi{\arabic{enumi}.}
\item
  E(cX)=cE(X) for any constant, c
\item
  E(X+c) = E(X) + c for any constant, c
\item
  Var(cX) = \(c^2\)Var(X). Note, this one makes more sense if you think
  about the standard deviation, which is the square root of the
  variance. It says that the standard deviation of cX is
  \textbar c\textbar{} times the standard deviation of X.
\item
  Var(X + c) = Var(X)
\end{enumerate}

These rules actually make sense if you think about an example. Suppose
we are playing a board game in which you roll a single die. Let X
represent the value rolled. We know that \(E[X]=3.5\).

\hypertarget{problem-1}{%
\subsubsection{Problem 1}\label{problem-1}}

\begin{enumerate}
\def\labelenumi{\alph{enumi})}
\item
  Suppose in this game, you go forward 2X spaces when you roll X. On
  average how many spaces forward do you go? (Which of the properties
  did you use?)
\item
  For the same rules as above, what is the standard deviation of the
  number of spaces you move ahead? Your answer can be expressed in terms
  of ``SD(X),'' the standard deviation of X. (Which of the properties
  did you use?)
\item
  Now suppose the rules are different. If we roll X, then we go forward
  X+4 spaces. What is the expected number of spaces that you go forward?
\item
  Using the rules in (c), what is the standard deviation for the number
  of spaces you go forward?
\end{enumerate}

One more property:

For any random variables X and Y,

E(X+Y) = E(X) + E(Y)

Suppose you roll two dice and X and Y are random variables representing
the values. What is the expected sum of the two dice?

\hypertarget{binomial-distribution}{%
\subsection{Binomial Distribution}\label{binomial-distribution}}

A binomial random variable, X\textasciitilde Binomial(n,p), is
characterized by the following:

\begin{enumerate}
\def\labelenumi{\arabic{enumi}.}
\item
  X = Number of ``successes'' out of n ``trials''
\item
  n is fixed in advance
\item
  The trials are independent of each other
\item
  For each trial there is a probability of success, p, that is the same
  for each trial
\end{enumerate}

Some examples:

\begin{enumerate}
\def\labelenumi{\arabic{enumi}.}
\tightlist
\item
  X = Number of hits in n ``at bats''
\item
  X = Number of cars with defective airbags out of n manufactured
\item
  X = Number of days you wake up on time in a week
\end{enumerate}

The probability mass function for X\textasciitilde Binomial(n,p) is

\[P(X=k) = \binom n k p^k(1-p)^{n-k}\]\\
for \(k\in\{0,1,2,...,n\}\). Here, \(\binom n k\) is the number of ways
to choose k items from n and is defined as

\(\binom n k\) \(:=\) \(\frac{n!}{k!(n-k)!}\)

The mean and variance of a binomial distribution are

\(E(X) = \mu = np\)

\(Var(X) = \sigma^2 = np(1-p)\)

\hypertarget{problem-2}{%
\subsubsection{Problem 2}\label{problem-2}}

The proportion of Norwegians in their early 40s who drink at least 1 cup
of coffee per day is about p = 0.894. Suppose that we take a random
sample of 50 Norwegians from this age group. Let X = the number of
Norwegians in the sample that drink at least 1 cup of coffee a day. In
your answers to the questions below, include the correct notation for
the probability being asked for in terms of the random variable X.

\begin{enumerate}
\def\labelenumi{\alph{enumi})}
\item
  What is the expected value of the number of Norwegians that drink at
  least 1 cup of coffee a day in the sample?
\item
  What is the standard deviation of the number of Norwegians that drink
  at least 1 cup of coffee a day in the sample?
\end{enumerate}

For questions 1.(c)-(e), calculate the required probabilities using R as
only a calculator. Include all formulas used.

\begin{enumerate}
\def\labelenumi{\alph{enumi})}
\setcounter{enumi}{2}
\tightlist
\item
  What is the probability that exactly 48 Norwegians in the sample drink
  at least 1 cup of coffee a day?
\end{enumerate}

\emph{\(P(X=48) = 50!/48!2!\) \(0.894^{48}(1-0.894)^{2}\) =}

\begin{enumerate}
\def\labelenumi{\alph{enumi})}
\setcounter{enumi}{3}
\item
  What is the probability that exactly 2 Norwegians in the sample do not
  drink at least 1 cup of coffee a day?
\item
  What is the probability that more than 2 Norwegians in the sample do
  not drink at least 1 cup of coffee a day?
\item
  The following simulates 10,000 observations from a Binomial(50,.894)
  distribution. \texttt{numCoffee} is the simulated observations. The
  last three lines of code use the simulated values to approximate the
  answers to (a)-(c).
\end{enumerate}

\begin{Shaded}
\begin{Highlighting}[]
\KeywordTok{set.seed}\NormalTok{(}\DecValTok{1}\NormalTok{)}
\NormalTok{num_simulations =}\StringTok{ }\DecValTok{10000}
\NormalTok{numCoffee =}\StringTok{ }\KeywordTok{rbinom}\NormalTok{(num_simulations, }\DataTypeTok{size=}\DecValTok{50}\NormalTok{, }\DataTypeTok{p=}\FloatTok{0.894}\NormalTok{)}
\CommentTok{# check a}
\KeywordTok{mean}\NormalTok{(numCoffee)}
\end{Highlighting}
\end{Shaded}

\begin{verbatim}
## [1] 44.6968
\end{verbatim}

\begin{Shaded}
\begin{Highlighting}[]
\CommentTok{# check b}
\KeywordTok{sd}\NormalTok{(numCoffee)}
\end{Highlighting}
\end{Shaded}

\begin{verbatim}
## [1] 2.194801
\end{verbatim}

\begin{Shaded}
\begin{Highlighting}[]
\CommentTok{# check c}
\KeywordTok{mean}\NormalTok{(numCoffee }\OperatorTok{==}\StringTok{ }\DecValTok{48}\NormalTok{)}
\end{Highlighting}
\end{Shaded}

\begin{verbatim}
## [1] 0.0644
\end{verbatim}

\hypertarget{problem-3}{%
\subsubsection{Problem 3}\label{problem-3}}

R has built in functions to calculate probabilities and quantiles for
the binomial distribution.

The \texttt{dbinom(k,n,p)} is the probability mass function for the
binomial distribution, i.e.~\(P(X=k)\) = \texttt{dbinom(k,n,p)}. For
this function, the following need to be specified:

\begin{itemize}
\item
  \texttt{k}, the number of successes associated with the desired
  probability
\item
  \texttt{n}, the number of trials
\item
  \texttt{p}, the probability of success for a single trial
\end{itemize}

The height of each bar in the following illustration is \(P(X=k)\) for
X\textasciitilde Binomial(10,.5) and k= 0,..,10.

\begin{Shaded}
\begin{Highlighting}[]
\KeywordTok{barplot}\NormalTok{(}\KeywordTok{dbinom}\NormalTok{(}\DecValTok{0}\OperatorTok{:}\DecValTok{10}\NormalTok{,}\DecValTok{10}\NormalTok{,.}\DecValTok{5}\NormalTok{),}\DataTypeTok{names.arg =} \DecValTok{0}\OperatorTok{:}\DecValTok{10}\NormalTok{, }\DataTypeTok{main=}\StringTok{"PMF of Binomial(10,0.5)"}\NormalTok{)}
\end{Highlighting}
\end{Shaded}

\includegraphics{lab5-3-_files/figure-latex/unnamed-chunk-9-1.pdf}

The \texttt{pbinom(k,n,p)} function in R is the probability distribution
function for the binomial distribution, i.e.~\(P(X \le k)\)
=\texttt{pbinom(k,n,p)}. Here \texttt{k}, \texttt{n}, and \texttt{p} are
as specified above.

The \texttt{pbinom(k,n,p)} function can also be used to find the upper
tail probabilities, i.e.~\(P(X>k)\), by using the optional parameter
\texttt{lower.tail=FALSE}. Thus, \(P(X>k)\) =
\texttt{pbinom(k,n,p,lower.tail=FALSE)}

For 3. (a)-(c), repeat the probability calculations for 2. (c)-(e) using
these R functions.

The \texttt{qbinom(p\_quant,n,p)} function in R is the quantile function
for the binomial distribution. It finds the smallest value of k such
that \(P(X \le k)\) \(\ge\) \texttt{p\_quant}. Here \texttt{k},
\texttt{n}, and \texttt{p} are as specified above. \texttt{p\_quant} is
such that the 100 \(\times\) \texttt{pquant} percentile of the binomial
distribution is given by \texttt{qbinom(p\_quant,n,p)}.For instance,
specifying \texttt{p\_quant\ =\ .5} will return the median of the
binomial distribution.

\begin{enumerate}
\def\labelenumi{\alph{enumi})}
\setcounter{enumi}{3}
\item
  Find the 25th and 75th percentiles of the binomial distribution
  specified in problem 2.
\item
  Find the smallest number such that the probability of the number of
  Norwegians that drink at least 1 cup of coffee a day in the sample is
  greater than this number is less than or equal to 10\%.
\end{enumerate}

\hypertarget{poisson-distribution}{%
\subsection{Poisson Distribution}\label{poisson-distribution}}

A poisson random variable, X\textasciitilde Poisson(\(\lambda\)), is
characterized by the following:

\begin{enumerate}
\def\labelenumi{\arabic{enumi}.}
\item
  X = Number of rare events that occur over a fixed amount of time or
  space
\item
  Events are independent
\item
  The maximum number of events that occur is not fixed
\end{enumerate}

Some examples:

\begin{enumerate}
\def\labelenumi{\arabic{enumi}.}
\item
  X = Number of tornadoes in a particular area over a year
\item
  X = Number of raindrops that fall on a particular square inch of roof
  during a one-second interval of time
\item
  X = Number of people that arrive at a train station during a 5 minute
  interval of time
\end{enumerate}

For X\textasciitilde Poisson(\(\lambda\)) and any \(k\in\{0,1,2,...\}\),
the following probability mass function defines P(X=k).

\(P(X=k)\) = \(e^{-\lambda}\lambda^{k}/k!\)

The mean and variance of a Poisson distribution are

\(E(X)\) = \(\mu = \lambda\)

\(Var(X)\) = \(\sigma^2 = \lambda\)

\hypertarget{problem-4}{%
\subsubsection{Problem 4}\label{problem-4}}

In the summer months, northern Minnesota has a thriving mosquito
population. When the average person steps outside in the summer, the
number of mosquito bites he receives every hour follows a Poisson(10)
distribution if he has not applied mosquito repellant. If he has applied
mosquito repellant, the number of mosquito bites he receives every hour
has a Poisson(3) distribution. Assume a randomly chosen Minnesotan has
just stepped outside without applying mosquito repellant. Let

X = number of mosquito bites this person receives in the first hour.

Y = number of mosquito bites the person receives in the second hour.

\begin{enumerate}
\def\labelenumi{\alph{enumi})}
\item
  What is the probability in the first hour outside this individual does
  not get bit by a mosquito?
\item
  What is \(P(10<X \leq 12)\)?
\item
  If this Minnesotan plans to stay outside for 2 hours in total and he
  applies mosquito repellant after the first hour, how many mosquito
  bites can he expect to receive?
\item
  After spending two hours outdoors, this Minnesotan will treat his
  mosquito bites with a topical ointment. For complete relief from the
  allergic reaction caused by the mosquito bites, the ointment must be
  applied to every mosquito bite 3 times. How many times in total will
  he expect to apply the ointment after the 2 hours to alleviate the
  allergic reaction of every bite?
\end{enumerate}

\hypertarget{problem-5}{%
\subsubsection{Problem 5}\label{problem-5}}

R also has built in functions to calculate probabilities and quantiles
for the Poisson distribution.

The \texttt{dpois}\((k,\lambda)\) is the probability mass function for
the Poisson distribution, i.e.~\(P(X=k)\) =
\texttt{dpois}\((k,\lambda)\). For this function, the following need to
be specified:

\begin{itemize}
\item
  \texttt{k}, the number of successes associated with the desired
  probability
\item
  \(\lambda\), the mean and variance of the Poisson distribution
\end{itemize}

The \texttt{ppois}\((k,\lambda)\) function in R is the cumulative
probability distribution function for the Poisson distribution,
i.e.~\(P(X \le k)\) = \texttt{ppois}\((k,\lambda)\). Here \texttt{k} and
\(\lambda\) are as specified above.

Similar to the \texttt{pbinom()} function, the
\texttt{ppois}\((k,\lambda)\) function can also be used to find the
upper tail probabilities, i.e.~\(P(X>k)\), by using the optional
parameter \texttt{lower.tail=FALSE}. Thus, \(P(X>k)\) =
\texttt{ppois(}k,\(\lambda\),\texttt{lower.tail=FALSE)}

For 5. (a)-(b), repeat the probability calculations for 4. (a)-(b) using
these R functions.

The \texttt{qpois}(p\_quant,\(\lambda\)) function in R is the quantile
function for the Poisson distribution. It finds the smallest value of k
such that \(P(X \le k)\) \(\ge\) \texttt{p\_quant}. Here \texttt{k} and
\(\lambda\) are as specified above. \texttt{p\_quant} is such that the
100 \(\times\) \texttt{pquant} percentile of the Poisson distribution is
given by \texttt{qpois}(p\_quant,\(\lambda\)).

\begin{enumerate}
\def\labelenumi{\alph{enumi})}
\setcounter{enumi}{2}
\tightlist
\item
  Find the median of the first Poisson distribution specified in problem
  4.
\end{enumerate}


\end{document}
