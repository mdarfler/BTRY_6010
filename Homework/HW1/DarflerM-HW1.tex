\documentclass[]{article}
\usepackage{lmodern}
\usepackage{amssymb,amsmath}
\usepackage{ifxetex,ifluatex}
\usepackage{fixltx2e} % provides \textsubscript
\ifnum 0\ifxetex 1\fi\ifluatex 1\fi=0 % if pdftex
  \usepackage[T1]{fontenc}
  \usepackage[utf8]{inputenc}
\else % if luatex or xelatex
  \ifxetex
    \usepackage{mathspec}
  \else
    \usepackage{fontspec}
  \fi
  \defaultfontfeatures{Ligatures=TeX,Scale=MatchLowercase}
\fi
% use upquote if available, for straight quotes in verbatim environments
\IfFileExists{upquote.sty}{\usepackage{upquote}}{}
% use microtype if available
\IfFileExists{microtype.sty}{%
\usepackage{microtype}
\UseMicrotypeSet[protrusion]{basicmath} % disable protrusion for tt fonts
}{}
\usepackage[margin=1in]{geometry}
\usepackage{hyperref}
\hypersetup{unicode=true,
            pdftitle={Homework 1 - File Management, R, RStudio, and LaTeX},
            pdfborder={0 0 0},
            breaklinks=true}
\urlstyle{same}  % don't use monospace font for urls
\usepackage{color}
\usepackage{fancyvrb}
\newcommand{\VerbBar}{|}
\newcommand{\VERB}{\Verb[commandchars=\\\{\}]}
\DefineVerbatimEnvironment{Highlighting}{Verbatim}{commandchars=\\\{\}}
% Add ',fontsize=\small' for more characters per line
\usepackage{framed}
\definecolor{shadecolor}{RGB}{248,248,248}
\newenvironment{Shaded}{\begin{snugshade}}{\end{snugshade}}
\newcommand{\AlertTok}[1]{\textcolor[rgb]{0.94,0.16,0.16}{#1}}
\newcommand{\AnnotationTok}[1]{\textcolor[rgb]{0.56,0.35,0.01}{\textbf{\textit{#1}}}}
\newcommand{\AttributeTok}[1]{\textcolor[rgb]{0.77,0.63,0.00}{#1}}
\newcommand{\BaseNTok}[1]{\textcolor[rgb]{0.00,0.00,0.81}{#1}}
\newcommand{\BuiltInTok}[1]{#1}
\newcommand{\CharTok}[1]{\textcolor[rgb]{0.31,0.60,0.02}{#1}}
\newcommand{\CommentTok}[1]{\textcolor[rgb]{0.56,0.35,0.01}{\textit{#1}}}
\newcommand{\CommentVarTok}[1]{\textcolor[rgb]{0.56,0.35,0.01}{\textbf{\textit{#1}}}}
\newcommand{\ConstantTok}[1]{\textcolor[rgb]{0.00,0.00,0.00}{#1}}
\newcommand{\ControlFlowTok}[1]{\textcolor[rgb]{0.13,0.29,0.53}{\textbf{#1}}}
\newcommand{\DataTypeTok}[1]{\textcolor[rgb]{0.13,0.29,0.53}{#1}}
\newcommand{\DecValTok}[1]{\textcolor[rgb]{0.00,0.00,0.81}{#1}}
\newcommand{\DocumentationTok}[1]{\textcolor[rgb]{0.56,0.35,0.01}{\textbf{\textit{#1}}}}
\newcommand{\ErrorTok}[1]{\textcolor[rgb]{0.64,0.00,0.00}{\textbf{#1}}}
\newcommand{\ExtensionTok}[1]{#1}
\newcommand{\FloatTok}[1]{\textcolor[rgb]{0.00,0.00,0.81}{#1}}
\newcommand{\FunctionTok}[1]{\textcolor[rgb]{0.00,0.00,0.00}{#1}}
\newcommand{\ImportTok}[1]{#1}
\newcommand{\InformationTok}[1]{\textcolor[rgb]{0.56,0.35,0.01}{\textbf{\textit{#1}}}}
\newcommand{\KeywordTok}[1]{\textcolor[rgb]{0.13,0.29,0.53}{\textbf{#1}}}
\newcommand{\NormalTok}[1]{#1}
\newcommand{\OperatorTok}[1]{\textcolor[rgb]{0.81,0.36,0.00}{\textbf{#1}}}
\newcommand{\OtherTok}[1]{\textcolor[rgb]{0.56,0.35,0.01}{#1}}
\newcommand{\PreprocessorTok}[1]{\textcolor[rgb]{0.56,0.35,0.01}{\textit{#1}}}
\newcommand{\RegionMarkerTok}[1]{#1}
\newcommand{\SpecialCharTok}[1]{\textcolor[rgb]{0.00,0.00,0.00}{#1}}
\newcommand{\SpecialStringTok}[1]{\textcolor[rgb]{0.31,0.60,0.02}{#1}}
\newcommand{\StringTok}[1]{\textcolor[rgb]{0.31,0.60,0.02}{#1}}
\newcommand{\VariableTok}[1]{\textcolor[rgb]{0.00,0.00,0.00}{#1}}
\newcommand{\VerbatimStringTok}[1]{\textcolor[rgb]{0.31,0.60,0.02}{#1}}
\newcommand{\WarningTok}[1]{\textcolor[rgb]{0.56,0.35,0.01}{\textbf{\textit{#1}}}}
\usepackage{graphicx,grffile}
\makeatletter
\def\maxwidth{\ifdim\Gin@nat@width>\linewidth\linewidth\else\Gin@nat@width\fi}
\def\maxheight{\ifdim\Gin@nat@height>\textheight\textheight\else\Gin@nat@height\fi}
\makeatother
% Scale images if necessary, so that they will not overflow the page
% margins by default, and it is still possible to overwrite the defaults
% using explicit options in \includegraphics[width, height, ...]{}
\setkeys{Gin}{width=\maxwidth,height=\maxheight,keepaspectratio}
\IfFileExists{parskip.sty}{%
\usepackage{parskip}
}{% else
\setlength{\parindent}{0pt}
\setlength{\parskip}{6pt plus 2pt minus 1pt}
}
\setlength{\emergencystretch}{3em}  % prevent overfull lines
\providecommand{\tightlist}{%
  \setlength{\itemsep}{0pt}\setlength{\parskip}{0pt}}
\setcounter{secnumdepth}{0}
% Redefines (sub)paragraphs to behave more like sections
\ifx\paragraph\undefined\else
\let\oldparagraph\paragraph
\renewcommand{\paragraph}[1]{\oldparagraph{#1}\mbox{}}
\fi
\ifx\subparagraph\undefined\else
\let\oldsubparagraph\subparagraph
\renewcommand{\subparagraph}[1]{\oldsubparagraph{#1}\mbox{}}
\fi

%%% Use protect on footnotes to avoid problems with footnotes in titles
\let\rmarkdownfootnote\footnote%
\def\footnote{\protect\rmarkdownfootnote}

%%% Change title format to be more compact
\usepackage{titling}

% Create subtitle command for use in maketitle
\providecommand{\subtitle}[1]{
  \posttitle{
    \begin{center}\large#1\end{center}
    }
}

\setlength{\droptitle}{-2em}

  \title{Homework 1 - File Management, R, RStudio, and LaTeX}
    \pretitle{\vspace{\droptitle}\centering\huge}
  \posttitle{\par}
    \author{}
    \preauthor{}\postauthor{}
    \date{}
    \predate{}\postdate{}
  

\begin{document}
\maketitle

\begin{center}\rule{0.5\linewidth}{\linethickness}\end{center}

\hypertarget{name-michael-darfler}{%
\section{NAME: Michael Darfler}\label{name-michael-darfler}}

\hypertarget{netid-mbd25}{%
\section{NETID: mbd25}\label{netid-mbd25}}

\hypertarget{due-date-september-7-2019-1159pm}{%
\section{\texorpdfstring{\textbf{DUE DATE: September 7, 2019,
11:59pm}}{DUE DATE: September 7, 2019, 11:59pm}}\label{due-date-september-7-2019-1159pm}}

\begin{center}\rule{0.5\linewidth}{\linethickness}\end{center}

\hypertarget{instructions}{%
\section{Instructions}\label{instructions}}

\begin{enumerate}
\def\labelenumi{\arabic{enumi})}
\setcounter{enumi}{-1}
\item
  Follow the instruction at
  \url{http://faculty.bscb.cornell.edu/~basu/BTRY6010/HW1_Getting_Started.html}
  to set up required softwares on your computer.
\item
  Read through this document and perform all indicated actions and
  answer all questions.
\item
  When you are finished, knit this document and upload the following
  \textbf{three} files to blackboard:

  \begin{enumerate}
  \def\labelenumii{\alph{enumii}.}
  \tightlist
  \item
    This modified .Rmd document
  \item
    The pdf that results after knitting this .Rmd file (and, if Word or
    html were used, converting to a pdf).
  \item
    The screenshot requested in
    \url{http://faculty.bscb.cornell.edu/~basu/BTRY6010/HW1_Getting_Started.html}.
  \end{enumerate}
\end{enumerate}

\hypertarget{code-chunks}{%
\section{Code Chunks}\label{code-chunks}}

RMarkdown documents (.Rmd) are designed for reproducible research. They
are composed of a combination of text and evaluated R code. Using R
Markdown, you can describe your study and perform the statistical
analysis in R in one document. This allows you to recreate this analysis
at any later point in time.

To run statistical analysis through R Markdown, you need to specify in
your .Rmd document where R can find the commands you would like to run.
Any R code you would like to run in your document must be in a
\emph{code chunk}. A code chunk looks like this:

To see the output of any code you put into a code chunk, you must knit
your document by selecting \emph{Knit (HTML or PDF or Word)} from the
menu for this document. If you knit your document right now, nothing
will show up for the above code chunk since there is no code to run in
it.

You will see several other code chunks throughout this document that
contain R code, but the code has not been evaluated (these are
highlighted in light gray). In this .Rmd document, the first line of
these code chunks are all set to \texttt{eval=FALSE}. For these code
chunks to be evaluated in your pdf or Word document, you need to set
\texttt{eval=TRUE}. For now, keep these code chunks set to
\texttt{eval=FALSE}.

\begin{enumerate}
\def\labelenumi{\arabic{enumi}.}
\tightlist
\item
  Add a code ``Chunk'' to this document here. To do this, first click on
  the place in this document where you want the code chunk to be added,
  then click on the arrow next to ``Chunks'' on the menu for this
  document and choose ``Insert Chunk''. \textbf{Important: Code chunks
  must be left justified to run the R code within properly.}
\end{enumerate}

In the next sections, we will explore some of the commands that can be
run in a code chunk.

\hypertarget{setting-your-working-directory}{%
\section{Setting Your Working
Directory}\label{setting-your-working-directory}}

The first thing you want to do when using R or Rstudio is to set your
``working directory''. R will assume that all of the files it needs to
access are in this directory. All .Rmd documents and the Console below
have \emph{separate} working directories. The following two R functions
associated with working directories can be used either in a code chunk
(for a .Rmd document) or in the console below.

\begin{enumerate}
\def\labelenumi{\arabic{enumi})}
\item
  \texttt{getwd()} - This command indicates your current working
  directory.
\item
  \texttt{setwd(\textquotesingle{}/filepath/\textquotesingle{})} - You
  can set the working directory for a .Rmd document or the console by
  specifying a filepath to the folder you would like to use as your
  working directory.
\end{enumerate}

\emph{For the console only} your working directory can also be set by
selecting the following from the RStudio menu at the top of your screen,
\emph{Session -\textgreater{} Set Working Directory -\textgreater{}
Choose Directory}.

\begin{enumerate}
\def\labelenumi{\arabic{enumi}.}
\item
  First we will determine and then set the working directory of the
  console.

  \begin{enumerate}
  \def\labelenumii{\alph{enumii}.}
  \item
    In the console below, type in: \texttt{getwd()} (without the
    tickmarks). This will show you the current working directory for the
    console.
  \item
    Change this working directory to the folder for homework 1 by either
    using the \texttt{setwd()} command or by using the menu options
    indicated above.
  \end{enumerate}
\item
  Now we will confirm that the working directory for this .Rmd document
  is the folder for homework 1 by using the \texttt{getwd()} command in
  a code chunk here. \textbf{Reminder: To see the output of any code you
  put into a code chunk, you must knit your .Rmd document.} Tell R to
  evaluate the following code in your knitted pdf/html/Word document by
  changing the first line in the code chunk given below to
  \texttt{eval=TRUE} (alternatively you can delete
  \texttt{,\ eval\ =\ FALSE}). Knit this document now to see the
  filepath of the working directory for your .Rmd document.
\end{enumerate}

\begin{Shaded}
\begin{Highlighting}[]
\KeywordTok{getwd}\NormalTok{()}
\end{Highlighting}
\end{Shaded}

\begin{verbatim}
## [1] "/Volumes/GoogleDrive/My Drive/BTRY 6010/HW1"
\end{verbatim}

\hypertarget{some-basic-r-commands}{%
\section{Some Basic R Commands}\label{some-basic-r-commands}}

\begin{enumerate}
\def\labelenumi{\arabic{enumi}.}
\tightlist
\item
  At the most basic level, R can be used as a calculator. The following
  chunk of code gives some basic R commands for calculation.
\end{enumerate}

\begin{Shaded}
\begin{Highlighting}[]
\DecValTok{5-3}
\end{Highlighting}
\end{Shaded}

\begin{verbatim}
## [1] 2
\end{verbatim}

\begin{Shaded}
\begin{Highlighting}[]
\DecValTok{5}\OperatorTok{+}\DecValTok{3}
\end{Highlighting}
\end{Shaded}

\begin{verbatim}
## [1] 8
\end{verbatim}

\begin{Shaded}
\begin{Highlighting}[]
\DecValTok{5}\OperatorTok{*}\DecValTok{3}
\end{Highlighting}
\end{Shaded}

\begin{verbatim}
## [1] 15
\end{verbatim}

\begin{Shaded}
\begin{Highlighting}[]
\DecValTok{5}\OperatorTok{/}\DecValTok{3}
\end{Highlighting}
\end{Shaded}

\begin{verbatim}
## [1] 1.666667
\end{verbatim}

\begin{Shaded}
\begin{Highlighting}[]
\DecValTok{5}\OperatorTok{^}\DecValTok{3}
\end{Highlighting}
\end{Shaded}

\begin{verbatim}
## [1] 125
\end{verbatim}

Now in the first line of this code chunk, set \texttt{eval=TRUE}. Knit
your pdf to see these mathematical expressions evaluated in your pdf
file.

For any code in your R Markdown document, you can also run it directly
in your R console by highlighting the code you wish to run and choosing
the ``Run'' option from the menu at the top of the .Rmd file. Do this
now with the code above. You will see your output in the console below.
This allows you to see how your code is working without having to knit a
Word document or pdf.

\begin{enumerate}
\def\labelenumi{\arabic{enumi}.}
\setcounter{enumi}{1}
\tightlist
\item
  You can use variables to represent various kinds of objects (numbers,
  vectors, matrices, function output, and data frames, which we will
  explain later) in R.
\end{enumerate}

The assignment operator in R is ``\textless{}-'' or ``=''. It assigns
whatever is on the left hand side to have the value of whatever is on
the right hand side. In the following data chunk, we will assign the
variable x to represent the number 5 and the variable y to represent the
number 3. We will now run the previous calculations using the variables
x and y.

\begin{Shaded}
\begin{Highlighting}[]
\NormalTok{x=}\DecValTok{5}
\NormalTok{y=}\DecValTok{3}
\NormalTok{x}\OperatorTok{-}\NormalTok{y}
\end{Highlighting}
\end{Shaded}

\begin{verbatim}
## [1] 2
\end{verbatim}

\begin{Shaded}
\begin{Highlighting}[]
\NormalTok{x}\OperatorTok{+}\NormalTok{y}
\end{Highlighting}
\end{Shaded}

\begin{verbatim}
## [1] 8
\end{verbatim}

\begin{Shaded}
\begin{Highlighting}[]
\NormalTok{x}\OperatorTok{*}\NormalTok{y}
\end{Highlighting}
\end{Shaded}

\begin{verbatim}
## [1] 15
\end{verbatim}

\begin{Shaded}
\begin{Highlighting}[]
\NormalTok{x}\OperatorTok{/}\NormalTok{y}
\end{Highlighting}
\end{Shaded}

\begin{verbatim}
## [1] 1.666667
\end{verbatim}

\begin{Shaded}
\begin{Highlighting}[]
\NormalTok{x}\OperatorTok{^}\NormalTok{y}
\end{Highlighting}
\end{Shaded}

\begin{verbatim}
## [1] 125
\end{verbatim}

Change \texttt{eval=TRUE} so that this code will run in your pdf or word
file.

You don't need to use just one letter as can be seen by running the code
below.

\begin{Shaded}
\begin{Highlighting}[]
\NormalTok{TOM=x}
\NormalTok{JERRY=y}
\NormalTok{TOM}\OperatorTok{+}\NormalTok{JERRY}
\end{Highlighting}
\end{Shaded}

\begin{verbatim}
## [1] 8
\end{verbatim}

However, R is case sensitive, so you will see an error when R tries to
run the code \texttt{Tom+Jerry} below. Set \texttt{eval=TRUE}. Two
things happen when you try to knit your pdf: 1) you will not get an
updated pdf and 2) an error message will pop up in the R Markdown tab of
RStudio below. The important information from this error message is that
the ``object `Tom' is not found.'' Now, set \texttt{eval=FALSE} so that
you can again knit this .Rmd document.

\begin{Shaded}
\begin{Highlighting}[]
\NormalTok{Tom}\OperatorTok{+}\NormalTok{Jerry}
\end{Highlighting}
\end{Shaded}

Note, unless you clear your workspace, R ``remembers'' any previously
defined variables.

\begin{enumerate}
\def\labelenumi{\arabic{enumi}.}
\setcounter{enumi}{2}
\tightlist
\item
  All of your R Markdown documents and the R Console below have separate
  ``workspaces.'' Your workspace contains all of the variables you have
  defined in your R Markdown document or your console with their values.
\end{enumerate}

Any variables that we defined in this R Markdown document that were not
run through the R Console below are in the workspace for this .Rmd
document, but not the workspace for the R Console. The workspace of your
R Console can be found under the ``Environment'' tab in the upper right
window of RStudio. Currently, this workspace should be empty since all
code where we have assigned values to variables was run only by knitting
a pdf from the .Rmd document and not directly through the R Console.

R code can be run directly through the R Console in three ways:

\begin{enumerate}
\def\labelenumi{\arabic{enumi})}
\item
  You can copy code from another document and paste it at the command
  prompt in your R Console.
\item
  As mentioned above, you can highlight the code in your R Markdown
  document and choose ``Run'' from the menu for that document.
\item
  You can type code in directly at the command prompt.
\end{enumerate}

Variables defined using any of these three methods will then be in the
workspace for the R Console.

Type \texttt{BUGSBUNNY=2} into the console below. \texttt{BUGSBUNNY} is
now in the workspace for the R Console. Check to make sure you now see
\texttt{BUGSBUNNY} defined under the ``Environment'' tab.

Notice \texttt{x} and \texttt{y} are not in the workspace for the R
Console, they are only in the workspace of this R Markdown document.
Type \texttt{x} in the console below. You should get the error message,
``Error: object `x' not found.'' You can put \texttt{x} and \texttt{y}
in your console's workspace by either copying the first chunk of code in
part (2) and pasting it at the command prompt in your console below OR
by highlighting this code and choosing ``Run'' from the menu above.
Please do this now. Note that \texttt{x} and \texttt{y} are now in your
workspace for the R console.

Now try and knit your pdf with the following code chunk set to
\texttt{eval=TRUE}.

\begin{Shaded}
\begin{Highlighting}[]
\NormalTok{BUGSBUNNY =}\StringTok{ }\DecValTok{2}
\NormalTok{BUGSBUNNY}\OperatorTok{+}\DecValTok{1}
\end{Highlighting}
\end{Shaded}

\begin{verbatim}
## [1] 3
\end{verbatim}

Again, you should have generated an error message in the R Markdown tab
below, and you should not have an updated pdf or Word document.
Although, \texttt{BUGSBUNNY} is defined in your R Console workspace, it
is not defined in this R Markdown document.

Edit the code above by including the line \texttt{BUGSBUNNY=2} above
\texttt{BUGSBUNNY+1}. Try to knit your document again.

Since you have now defined \texttt{BUGSBUNNY} in the .Rmd document, this
code should be evaluated in your pdf or Word document without error.

Every time you start a new project, you will want to clear your
workspace in the R Console. To clear your workspace at any time, from
the RStudio menu choose \emph{Session -\textgreater{} Clear Workspace}.

\begin{enumerate}
\def\labelenumi{\arabic{enumi}.}
\setcounter{enumi}{3}
\tightlist
\item
  For any function in R for which you currently have the library for
  that function installed and loaded, all documentation for that
  function can be found by typing a question mark in front of the name
  of the function and running it in the R console. For instance, to get
  documentation on the \texttt{getwd()} function, run the following R
  code by highlighting it and selecting ``Run'' from the menu above.
\end{enumerate}

\begin{Shaded}
\begin{Highlighting}[]
\NormalTok{?getwd}
\end{Highlighting}
\end{Shaded}

The documentation for \texttt{getwd} will now appear in the lower right
window of Rstudio under the tab ``Help.''

\hypertarget{some-general-comments-on-coding-and-debugging}{%
\section{Some General Comments on Coding and
Debugging}\label{some-general-comments-on-coding-and-debugging}}

Every programming language has a proper syntax that it understands. When
you do not use this syntax properly, one of the following will happen:

\begin{enumerate}
\def\labelenumi{\arabic{enumi})}
\item
  Your code will not run, and you will receive an error message.
\item
  Your code will run, but not as you intended it to.
\end{enumerate}

When you use a R Markdown document, you are working with two different
computer languages: one that tells R Markdown how to knit the document
and the other to evaluate all code chunks in the document (R). Keep the
following in mind when you are editing R Markdown documents:

\begin{enumerate}
\def\labelenumi{\arabic{enumi})}
\item
  Before you edit your R Markdown document for a given homework, knit
  it. This will give you a starting point from which you know that your
  document could knit correctly.
\item
  You do not need to knit your document every time you edit the R
  Markdown file, but you should knit often enough that it narrows down
  the area in which you have a coding error.
\item
  How things are spaced is important in R Markdown. If your file is not
  knitting \emph{nicely}, you may have an issue with spacing.
\item
  R Markdown will not knit \emph{copies}. So, if your R markdown file
  has a (1) or (2) before .Rmd (e.g.~BienJ-HW1(1).Rmd), you need to
  re-name it with the (1) or (2) removed and then knit.
\item
  Don't panic if you have a coding error either in R or R Markdown.

  \begin{enumerate}
  \def\labelenumii{\alph{enumii})}
  \item
    Your first step should be to look at the error message generated
    when you tried to run your code or knit your document. Often (but
    not always) the error message will tell you exactly what you need to
    fix. If you don't understand the error message, copy and paste the
    error message into google and see if something useful comes up.
    Google results from stackoverflow are particularly helpful for
    coding errors.
  \item
    If (a) doesn't help:

    \begin{enumerate}
    \def\labelenumiii{\roman{enumiii}.}
    \item
      Re-save your file under a different file name.
    \item
      Find the original un-edited .Rmd file in your ``Downloads''
      folder. Move it to the folder you created for the current homework
      and re-name it as specified for the current homework
      (e.g.~BienJ-HW1.Rmd).
    \item
      Copy and paste your code/answers from your file from (1) one by
      one into the new file. Knit every couple of edits to confirm your
      document is still knitting.\\
    \item
      If your document doesn't knit after a specific edit, your error is
      in this piece of code.
    \end{enumerate}
  \end{enumerate}
\item
  \textbf{How to ask for coding help from TA's}: If you have a bug in
  your code and are not able to fix it, first try to have the following
  information

  \begin{enumerate}
  \def\labelenumii{\alph{enumii})}
  \item
    Have a good idea of where the error in your code is e.g.~by going
    through the steps in 5b.
  \item
    Take a screenshot of the error that pops up while trying to knit
    your code. If you can get a screenshot of an error while running the
    code in the console, this could potentially be more informative.
  \end{enumerate}

  If you know there is a specific line of your code that is causing the
  error, \textbf{and the code does not give away a homework solution}
  you can post your code + error on Piazza. Otherwise you can send an
  email to TA, cc-ing the Instructor. Please include your .rmd file, a
  short description of where you believe the error is occurring, and a
  screenshot of the error you are receiving.
\end{enumerate}


\end{document}
