\documentclass[]{article}
\usepackage{lmodern}
\usepackage{amssymb,amsmath}
\usepackage{ifxetex,ifluatex}
\usepackage{fixltx2e} % provides \textsubscript
\ifnum 0\ifxetex 1\fi\ifluatex 1\fi=0 % if pdftex
  \usepackage[T1]{fontenc}
  \usepackage[utf8]{inputenc}
\else % if luatex or xelatex
  \ifxetex
    \usepackage{mathspec}
  \else
    \usepackage{fontspec}
  \fi
  \defaultfontfeatures{Ligatures=TeX,Scale=MatchLowercase}
\fi
% use upquote if available, for straight quotes in verbatim environments
\IfFileExists{upquote.sty}{\usepackage{upquote}}{}
% use microtype if available
\IfFileExists{microtype.sty}{%
\usepackage{microtype}
\UseMicrotypeSet[protrusion]{basicmath} % disable protrusion for tt fonts
}{}
\usepackage[margin=1in]{geometry}
\usepackage{hyperref}
\hypersetup{unicode=true,
            pdftitle={Homework 5: Uniform and Normal Distributions, the Central Limit Theorem (CLT)},
            pdfborder={0 0 0},
            breaklinks=true}
\urlstyle{same}  % don't use monospace font for urls
\usepackage{color}
\usepackage{fancyvrb}
\newcommand{\VerbBar}{|}
\newcommand{\VERB}{\Verb[commandchars=\\\{\}]}
\DefineVerbatimEnvironment{Highlighting}{Verbatim}{commandchars=\\\{\}}
% Add ',fontsize=\small' for more characters per line
\usepackage{framed}
\definecolor{shadecolor}{RGB}{248,248,248}
\newenvironment{Shaded}{\begin{snugshade}}{\end{snugshade}}
\newcommand{\AlertTok}[1]{\textcolor[rgb]{0.94,0.16,0.16}{#1}}
\newcommand{\AnnotationTok}[1]{\textcolor[rgb]{0.56,0.35,0.01}{\textbf{\textit{#1}}}}
\newcommand{\AttributeTok}[1]{\textcolor[rgb]{0.77,0.63,0.00}{#1}}
\newcommand{\BaseNTok}[1]{\textcolor[rgb]{0.00,0.00,0.81}{#1}}
\newcommand{\BuiltInTok}[1]{#1}
\newcommand{\CharTok}[1]{\textcolor[rgb]{0.31,0.60,0.02}{#1}}
\newcommand{\CommentTok}[1]{\textcolor[rgb]{0.56,0.35,0.01}{\textit{#1}}}
\newcommand{\CommentVarTok}[1]{\textcolor[rgb]{0.56,0.35,0.01}{\textbf{\textit{#1}}}}
\newcommand{\ConstantTok}[1]{\textcolor[rgb]{0.00,0.00,0.00}{#1}}
\newcommand{\ControlFlowTok}[1]{\textcolor[rgb]{0.13,0.29,0.53}{\textbf{#1}}}
\newcommand{\DataTypeTok}[1]{\textcolor[rgb]{0.13,0.29,0.53}{#1}}
\newcommand{\DecValTok}[1]{\textcolor[rgb]{0.00,0.00,0.81}{#1}}
\newcommand{\DocumentationTok}[1]{\textcolor[rgb]{0.56,0.35,0.01}{\textbf{\textit{#1}}}}
\newcommand{\ErrorTok}[1]{\textcolor[rgb]{0.64,0.00,0.00}{\textbf{#1}}}
\newcommand{\ExtensionTok}[1]{#1}
\newcommand{\FloatTok}[1]{\textcolor[rgb]{0.00,0.00,0.81}{#1}}
\newcommand{\FunctionTok}[1]{\textcolor[rgb]{0.00,0.00,0.00}{#1}}
\newcommand{\ImportTok}[1]{#1}
\newcommand{\InformationTok}[1]{\textcolor[rgb]{0.56,0.35,0.01}{\textbf{\textit{#1}}}}
\newcommand{\KeywordTok}[1]{\textcolor[rgb]{0.13,0.29,0.53}{\textbf{#1}}}
\newcommand{\NormalTok}[1]{#1}
\newcommand{\OperatorTok}[1]{\textcolor[rgb]{0.81,0.36,0.00}{\textbf{#1}}}
\newcommand{\OtherTok}[1]{\textcolor[rgb]{0.56,0.35,0.01}{#1}}
\newcommand{\PreprocessorTok}[1]{\textcolor[rgb]{0.56,0.35,0.01}{\textit{#1}}}
\newcommand{\RegionMarkerTok}[1]{#1}
\newcommand{\SpecialCharTok}[1]{\textcolor[rgb]{0.00,0.00,0.00}{#1}}
\newcommand{\SpecialStringTok}[1]{\textcolor[rgb]{0.31,0.60,0.02}{#1}}
\newcommand{\StringTok}[1]{\textcolor[rgb]{0.31,0.60,0.02}{#1}}
\newcommand{\VariableTok}[1]{\textcolor[rgb]{0.00,0.00,0.00}{#1}}
\newcommand{\VerbatimStringTok}[1]{\textcolor[rgb]{0.31,0.60,0.02}{#1}}
\newcommand{\WarningTok}[1]{\textcolor[rgb]{0.56,0.35,0.01}{\textbf{\textit{#1}}}}
\usepackage{graphicx,grffile}
\makeatletter
\def\maxwidth{\ifdim\Gin@nat@width>\linewidth\linewidth\else\Gin@nat@width\fi}
\def\maxheight{\ifdim\Gin@nat@height>\textheight\textheight\else\Gin@nat@height\fi}
\makeatother
% Scale images if necessary, so that they will not overflow the page
% margins by default, and it is still possible to overwrite the defaults
% using explicit options in \includegraphics[width, height, ...]{}
\setkeys{Gin}{width=\maxwidth,height=\maxheight,keepaspectratio}
\IfFileExists{parskip.sty}{%
\usepackage{parskip}
}{% else
\setlength{\parindent}{0pt}
\setlength{\parskip}{6pt plus 2pt minus 1pt}
}
\setlength{\emergencystretch}{3em}  % prevent overfull lines
\providecommand{\tightlist}{%
  \setlength{\itemsep}{0pt}\setlength{\parskip}{0pt}}
\setcounter{secnumdepth}{0}
% Redefines (sub)paragraphs to behave more like sections
\ifx\paragraph\undefined\else
\let\oldparagraph\paragraph
\renewcommand{\paragraph}[1]{\oldparagraph{#1}\mbox{}}
\fi
\ifx\subparagraph\undefined\else
\let\oldsubparagraph\subparagraph
\renewcommand{\subparagraph}[1]{\oldsubparagraph{#1}\mbox{}}
\fi

%%% Use protect on footnotes to avoid problems with footnotes in titles
\let\rmarkdownfootnote\footnote%
\def\footnote{\protect\rmarkdownfootnote}

%%% Change title format to be more compact
\usepackage{titling}

% Create subtitle command for use in maketitle
\providecommand{\subtitle}[1]{
  \posttitle{
    \begin{center}\large#1\end{center}
    }
}

\setlength{\droptitle}{-2em}

  \title{Homework 5: Uniform and Normal Distributions, the Central Limit Theorem
(CLT)}
    \pretitle{\vspace{\droptitle}\centering\huge}
  \posttitle{\par}
    \author{}
    \preauthor{}\postauthor{}
    \date{}
    \predate{}\postdate{}
  

\begin{document}
\maketitle

\begin{center}\rule{0.5\linewidth}{\linethickness}\end{center}

\hypertarget{name-your-name}{%
\section{NAME: Your Name}\label{name-your-name}}

\hypertarget{netid-your-netid}{%
\section{NETID: Your NetID}\label{netid-your-netid}}

\textbf{DUE DATE: October 12, 2019 by 11:59pm}

\begin{center}\rule{0.5\linewidth}{\linethickness}\end{center}

\textbf{For this homework, it will be helpful to have a copy of the
knitted version of this document to answer the questions as much of it
is written using mathematical notation that may be difficult to read
when the document is not knitted.}

\hypertarget{instructions}{%
\subsection{Instructions}\label{instructions}}

For this homework:

\begin{enumerate}
\def\labelenumi{\arabic{enumi}.}
\item
  All calculations must be done within your document in code chunks.
  Provide all intermediate steps.
\item
  Include any mathematical formulas you are using for a calculation.
  Surrounding mathematical expresses by dollar signs makes the math look
  nicer and lets you use a special syntax (called latex) that allows for
  Greek letters, fractions, etc. Note that this is not R code and
  therefore should not be put in a code chunk. You can put these
  immediately before the code chunk where you actually do the
  calculation.
\end{enumerate}

\hypertarget{some-notation}{%
\subsubsection{Some Notation}\label{some-notation}}

Your solutions to the problems below must include the formula used for
each calculation. To get you started, here is some mathematical
expressions written in latex that you may find helpful when writing out
the math in your answers. You can copy, paste, and edit these
expressions as needed.

For \(X \sim N(\mu, \sigma)\) and real numbers \(a\) and \(b\):

\begin{enumerate}
\def\labelenumi{\arabic{enumi})}
\tightlist
\item
  \(P(X \leq b) = P(Z \leq (b-\mu)/\sigma)\)
\item
  \(P(X \geq a) = P(Z \geq (a-\mu)/\sigma)\)
\item
  \(P(a \leq X \leq b) = P((a-\mu)/\sigma \leq Z \leq (b-\mu)/\sigma)\)
\end{enumerate}

\begin{center}\rule{0.5\linewidth}{\linethickness}\end{center}

\textbf{In this homework we will explore two continuous distributions
(uniform and normal) and the Central Limit Theorem. For uniform
distribution, probability calculation often reduces to calculating areas
of rectangles. For normal distribution, one needs to use z-scores along
with normal probability tables (see Appendix B of textbook). Here is a
brief review of normal distribution and z-scores. }

For \(X\sim N(\mu,\sigma)\) and an interval \((a,b)\) on the real line,

\[P(X \in (a,b)) = \int_{a}^{b} \frac{1}{\sqrt{2\pi\sigma^2}}e^{-\frac{1}{2\sigma^2}(x-\mu)^2}dx\]

(i.e., area under the pdf between \(a\) and \(b\)). As noted in lecture,
this cannot be computed in closed form; however, in R this can be
computed numerically.

For \(X \sim N(\mu, \sigma)\), the probability of getting a value in any
interval on the real line can be expressed solely in terms of the
cumulative distribution function, \(P(X \leq x)\). For any real numbers,
\(a\) and \(b\), \(a < b\):

\begin{enumerate}
\def\labelenumi{\arabic{enumi}.}
\item
  \(P(X < b)\) = \(P(X \leq b)\), the probability of getting a value
  less than (or equal to) \(b\)
\item
  \(P(X > a)\) = \(P(X \geq a)\) = \(1 - P(X < a)\), the probability of
  getting a value greater than (or equal to) a
\item
  \(P(a < X < b)\) = \(P(X < b) - P(X < a)\), the probability of getting
  a value between a and b
\item
  \(P(X = a) = 0\), the probability of getting the value a
\end{enumerate}

In R, the \texttt{pnorm}(x,\(\mu\),\(\sigma\)) function is the
cumulative probability distribution function for the normal distribution
with mean, \(\mu\) and standard deviation, \(\sigma\), evaluated at x,
i.e.~\(P(X \leq x)\) =\texttt{pnorm}(x,\(\mu\),\(\sigma\)) for
\(X \sim N(\mu,\sigma)\).

\hypertarget{calculating-probabilities-associated-with-the-normal-distribution-using-z-scores}{%
\subsubsection{Calculating Probabilities Associated with The Normal
Distribution Using
z-scores}\label{calculating-probabilities-associated-with-the-normal-distribution-using-z-scores}}

Every normal distribution, \(N(\mu,\sigma)\), can be seen as a shifted
and scaled standard normal distribution, \(N(0,1)\).

Assume, \(X \sim N(\mu,\sigma)\) and \(Z \sim N(0,1)\). Then

\(\frac{X-\mu}{\sigma} \sim Z\).

Thus, every quantile in the sample space of \(X\) has a corresponding
``standardized'' quantile in the sample space of \(Z\) (called the
z-score). If \(b\) is an outcome in the sample space of \(X\), the
corresponding standardized value of \(b\) in the sample space of \(Z\)
is

\(\frac{b-\mu}{\sigma}\) = z-score for \(b\).

Using z-scores, probabilities for \(X\) can be determined by
transforming each quantile of \(X\) into a standardized quantile and
using the probability distribution function for the standard normal
distribution. For example, for any real \(b\),

\(P(X \leq b) = P(Z \leq \frac{b-\mu}{\sigma})\)

In R, the \texttt{pnorm(x)} function without a mean or standard
deviation specified is the cumulative probability distribution function
for the standard normal distribution evaluated at x,
i.e.~\(P(Z \leq z)\) =\texttt{pnorm(z)}.

\hypertarget{problem-1}{%
\subsubsection{Problem 1}\label{problem-1}}

Let \(X\) denote the checkout time (in minutes) of a customer in a
Grocery store. Assume \(X\) is a random variable uniformly distributed
between \(10\) and \(20\) minutes, i.e., \(X \sim Unif(10, 20)\).

\begin{enumerate}
\def\labelenumi{\alph{enumi})}
\tightlist
\item
  Write down the probability density function (pdf) of \(X\), by
  replacing the question marks (?) with appropriate values in the next
  two lines. \[
  f(x) = ? \mbox{ for }  10 \le x \le 20
  \] and
\end{enumerate}

\[
f(x) = ? \mbox{ for } x < 10 \mbox{ or } x > 20.
\]

\begin{enumerate}
\def\labelenumi{\alph{enumi})}
\setcounter{enumi}{1}
\item
  What is the expected checkout time?
\item
  What is the first quartile (\(Q_1\)) of checkout times? Show your
  calculations.
\item
  What is the chance that a randomly selected customer at that store
  will have to wait more than \(15\) minutes? Show your calculations.
\item
  Now answer the question in d) using R function \texttt{punif} (Type
  \texttt{help(punif)} on the R console to know more about this
  function). Insert a code chunk below that uses \texttt{punif} to
  calculate this probability. Make sure to set \texttt{echo=TRUE} and
  \texttt{eval=TRUE} at the start of your code chunk.
\end{enumerate}

\hypertarget{problem-2}{%
\subsubsection{Problem 2}\label{problem-2}}

The daily milk production of a Guernsey cow has a normal distribution
with \(\mu = 70\) pounds and \(\sigma = 13\). A Guernsey cow is chosen
at random. Let \(X\) = Milk production in one day for a Guernsey cow.

\textbf{For (a) - (c) answer each question in two ways:}

\textbf{1) Using the \texttt{pnorm()} function with the mean and
standard deviation for X specified.}

\textbf{2) By converting all probabilities in terms of the standard
normal distribution and using the \texttt{pnorm()} function without the
mean and standard deviation specified.}

\textbf{For both (1) and (2), the formula you are using to calculate
each probability must be included before the code chunk where the answer
is evaluated.}

\begin{enumerate}
\def\labelenumi{\alph{enumi})}
\item
  What is the probability that a Guernsey cow chosen at random produces
  more than 90 pounds of milk in a given day?
\item
  What is the probability that a Guernsey cow chosen at random produces
  between 85 and 100 pounds of milk in a given day?
\item
  What is the probability that the quantity of milk produced by a
  Guernsey cow chosen at random is within 1.5 standard deviations of the
  mean number of pounds of milk produced by a Guernsey cow?
\end{enumerate}

\hypertarget{problem-3}{%
\subsubsection{Problem 3}\label{problem-3}}

As we have seen for other probability calculations, we can simulate from
a \(N(70, 13)\) distribution to estimate the probabilities computed
above in Problem 2.

The \texttt{rnorm}(n,\(\mu\),\(\sigma\)) function in R will simulate n
draws from a normal distribution with mean, \(\mu\) and standard
deviation, \(\sigma\).

\begin{enumerate}
\def\labelenumi{\roman{enumi})}
\tightlist
\item
  Include a code chunk here that simulates 10,000 draws from the N(70,
  13) distribution to estimate the probability computed in 2c). As in
  previous code for estimating probabilities through simulation, you
  will need to count the number of simulated draws that meet the
  criterion associated with the event defined in 2c) to be able to
  estimate this probability.
\end{enumerate}

\hypertarget{another-way-to-simulate-random-draws-from-n7013}{%
\subsubsection{\texorpdfstring{Another Way to Simulate Random Draws from
\(N(70,13)\)}{Another Way to Simulate Random Draws from N(70,13)}}\label{another-way-to-simulate-random-draws-from-n7013}}

Randomly drawing from \(X \sim N(70, 13)\) using \texttt{rnorm(n,70,13)}
is equivalent to:

\begin{enumerate}
\def\labelenumi{\arabic{enumi})}
\tightlist
\item
  Drawing randomly from a N(0,1) distribution
\item
  Multiplying (1) by the standard deviation of X
\item
  Adding the mean of X to (2)
\end{enumerate}

So, instead of using \texttt{rnorm(n,70,13)} to draw from
\(X \sim N(70,13)\), you can also use 13 \(\times\) \texttt{rnorm(n)} +
70 to draw from \(N(70, 13)\).

\begin{enumerate}
\def\labelenumi{\roman{enumi})}
\setcounter{enumi}{1}
\tightlist
\item
  Include a code chunk here that estimates the probability of 2c)
  through simulating 10,000 draws from the standard normal distribution
  and using the criterion for counting events associated with 2c)
  determined in (i).
\end{enumerate}

\hypertarget{problem-4}{%
\subsubsection{Problem 4}\label{problem-4}}

Here we will take a look at how an entire probability density function
can be approximated by simulation.

\begin{enumerate}
\def\labelenumi{\roman{enumi})}
\item
  In a code chunk here simulate 10,000 draws from a \(N(2,3)\)
  distribution. Call these simulated values, \texttt{Sim\_N23}.
\item
  In another code chunk, create a probability histogram of
  \texttt{Sim\_N23} using \texttt{hist()}. Remember to set
  \texttt{freq=FALSE.} Set \texttt{breaks\ =\ 75} and
  \texttt{main=\ \textquotesingle{}10,000\ Simulated\ Draws\ from\ N(2,3)\textquotesingle{}}.
\item
  The R function \texttt{lines(x,y)} will add a line to an existing
  plot. The arguments, \texttt{x} and \texttt{y}, are vectors of x and y
  coordinates that together define all the points the line must run
  through. In the code chunk you used to create the histogram above, we
  will now add code to overlay the histogram with the probability
  density function for the \(N(2,3)\) distribution. You can do this in
  three lines of code:
\end{enumerate}

\begin{enumerate}
\def\labelenumi{\alph{enumi})}
\item
  Define the x coordinates as
  \texttt{xvalues=seq(-10,\ 15,\ length=150)}. \texttt{xvalues} will
  include an evenly spaced grid of x coordinates.
\item
  Assign the output of \texttt{dnorm(xvalues,\ 2,\ 3)} to a vector named
  \texttt{yvalues}.
\item
  Use the \texttt{lines()} function to overlay the histogram created in
  (ii) with the probability density function for \(N(2,3)\). In
  particular \texttt{lines(xvalues,\ yvalues)} should do the trick.
\end{enumerate}

\hypertarget{problem-5}{%
\subsubsection{Problem 5}\label{problem-5}}

In lecture we saw that the sampling distribution of the sample mean when
you draw simple random samples seemed to look like a normal. This is a
general phenomenon, known as the Central Limit Theorem. In particular,
if \(X_i, i = 1, \ldots, n\) is an independent random sample from just
about \emph{any} distribution with mean, \(\mu\) and standard deviation
\(\sigma\), then for large enough \(n\),

\hypertarget{bar-x_n-approx-nmu-sigmasqrtn.}{%
\subsection{\texorpdfstring{\(\bar X_n \approx N(\mu, \sigma/\sqrt{n})\).}{\textbackslash bar X\_n \textbackslash approx N(\textbackslash mu, \textbackslash sigma/\textbackslash sqrt\{n\}).}}\label{bar-x_n-approx-nmu-sigmasqrtn.}}

Here we will illustrate the Central Limit Theorem by simulating
independent draws from the X \(\sim\) Poisson(\(\lambda\) = .5)
distribution.

\begin{enumerate}
\def\labelenumi{\alph{enumi})}
\tightlist
\item
  We start by looking at the PMF of a Poisson(0.5). Change the code from
  \texttt{eval=FALSE} to \texttt{eval=TRUE} once you make sure you
  understand every line of the code chunk.
\end{enumerate}

\begin{Shaded}
\begin{Highlighting}[]
\NormalTok{probs =}\StringTok{ }\KeywordTok{dpois}\NormalTok{(}\KeywordTok{c}\NormalTok{(}\DecValTok{0}\OperatorTok{:}\DecValTok{6}\NormalTok{), }\FloatTok{0.5}\NormalTok{)}
\KeywordTok{barplot}\NormalTok{(probs,}\DataTypeTok{names.arg =} \KeywordTok{c}\NormalTok{(}\DecValTok{0}\NormalTok{,}\DecValTok{1}\NormalTok{,}\DecValTok{2}\NormalTok{,}\DecValTok{3}\NormalTok{,}\DecValTok{4}\NormalTok{,}\DecValTok{5}\NormalTok{,}\DecValTok{6}\NormalTok{),}\DataTypeTok{main=}\StringTok{'PMF for Poisson(0.5)'}\NormalTok{)}
\end{Highlighting}
\end{Shaded}

\begin{enumerate}
\def\labelenumi{\alph{enumi})}
\setcounter{enumi}{1}
\item
  Using one or two lines in a code chunk, generate a realization of the
  sample mean of 4 independent draws from Poisson(.5). You can generate
  the sample using, \texttt{rpois(4,.5)}.
\item
  To simulate draws from the sampling distribution of \(\bar X_4\), we
  want to repeat (b) many times, storing the results in a vector.
\end{enumerate}

\begin{enumerate}
\def\labelenumi{(\roman{enumi})}
\item
  In a code chunk, define the vector of sample means as
  \texttt{xbar.realizations\ =\ rep(NA,num.simulations)} where
  \texttt{num.simulations\ =\ 10000}.
\item
  In this code chunk, write a \texttt{for} loop that repeats part (b)
  10,000 times, storing the sample mean generated in iteration i as
  \texttt{xbar.realizations{[}i{]}}. \texttt{xbar.realizations} will now
  contain 10,000 draws from the sampling distribution of \(\bar X_4\).
\end{enumerate}

\begin{enumerate}
\def\labelenumi{\alph{enumi})}
\setcounter{enumi}{3}
\item
  Create a probability histogram of \texttt{xbar.realizations.} Set
  \texttt{main=expression(paste(\textquotesingle{}Histogram\ of\ 10,000\ Draws\ of\ \textquotesingle{},bar(X){[}n{]}))}.
\item
  Include a code chunk here that defines, \texttt{n=3}. In this code
  chunk, include code that repeats parts (b) - (d) using a random sample
  of \texttt{n} observations instead of 4. Also, as in Problem 4iii),
  overlay the probability histogram that is created in the last step
  with the probability density function for a \(N(0.5,\sqrt{0.5/n})\)
  distribution. To do so, use the following:
\end{enumerate}

\begin{Shaded}
\begin{Highlighting}[]
\NormalTok{xvalues =}\StringTok{ }\KeywordTok{seq}\NormalTok{(}\DecValTok{0}\NormalTok{,}\DecValTok{2}\NormalTok{,.}\DecValTok{01}\NormalTok{)}
\NormalTok{yvalues =}\StringTok{ }\KeywordTok{dnorm}\NormalTok{(xvalues,.}\DecValTok{5}\NormalTok{, }\KeywordTok{sqrt}\NormalTok{(}\FloatTok{0.5}\OperatorTok{/}\NormalTok{n))}
\KeywordTok{lines}\NormalTok{(xvalues,yvalues)}
\end{Highlighting}
\end{Shaded}

\begin{enumerate}
\def\labelenumi{\alph{enumi})}
\setcounter{enumi}{5}
\tightlist
\item
  Repeat part (e) three times. The first time, set \texttt{n=10}. The
  second time, set \texttt{n=30}. The third time, set \texttt{n=50}. Be
  sure you include the probability histogram overlaid with the pdf for
  \(N(0.5,\sqrt{0.5/n})\) for every choice of \texttt{n}.
\end{enumerate}


\end{document}
