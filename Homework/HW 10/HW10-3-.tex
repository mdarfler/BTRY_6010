\documentclass[]{article}
\usepackage{lmodern}
\usepackage{amssymb,amsmath}
\usepackage{ifxetex,ifluatex}
\usepackage{fixltx2e} % provides \textsubscript
\ifnum 0\ifxetex 1\fi\ifluatex 1\fi=0 % if pdftex
  \usepackage[T1]{fontenc}
  \usepackage[utf8]{inputenc}
\else % if luatex or xelatex
  \ifxetex
    \usepackage{mathspec}
  \else
    \usepackage{fontspec}
  \fi
  \defaultfontfeatures{Ligatures=TeX,Scale=MatchLowercase}
\fi
% use upquote if available, for straight quotes in verbatim environments
\IfFileExists{upquote.sty}{\usepackage{upquote}}{}
% use microtype if available
\IfFileExists{microtype.sty}{%
\usepackage{microtype}
\UseMicrotypeSet[protrusion]{basicmath} % disable protrusion for tt fonts
}{}
\usepackage[margin=1in]{geometry}
\usepackage{hyperref}
\hypersetup{unicode=true,
            pdftitle={Homework 10: Multiple Linear Regression},
            pdfborder={0 0 0},
            breaklinks=true}
\urlstyle{same}  % don't use monospace font for urls
\usepackage{color}
\usepackage{fancyvrb}
\newcommand{\VerbBar}{|}
\newcommand{\VERB}{\Verb[commandchars=\\\{\}]}
\DefineVerbatimEnvironment{Highlighting}{Verbatim}{commandchars=\\\{\}}
% Add ',fontsize=\small' for more characters per line
\usepackage{framed}
\definecolor{shadecolor}{RGB}{248,248,248}
\newenvironment{Shaded}{\begin{snugshade}}{\end{snugshade}}
\newcommand{\AlertTok}[1]{\textcolor[rgb]{0.94,0.16,0.16}{#1}}
\newcommand{\AnnotationTok}[1]{\textcolor[rgb]{0.56,0.35,0.01}{\textbf{\textit{#1}}}}
\newcommand{\AttributeTok}[1]{\textcolor[rgb]{0.77,0.63,0.00}{#1}}
\newcommand{\BaseNTok}[1]{\textcolor[rgb]{0.00,0.00,0.81}{#1}}
\newcommand{\BuiltInTok}[1]{#1}
\newcommand{\CharTok}[1]{\textcolor[rgb]{0.31,0.60,0.02}{#1}}
\newcommand{\CommentTok}[1]{\textcolor[rgb]{0.56,0.35,0.01}{\textit{#1}}}
\newcommand{\CommentVarTok}[1]{\textcolor[rgb]{0.56,0.35,0.01}{\textbf{\textit{#1}}}}
\newcommand{\ConstantTok}[1]{\textcolor[rgb]{0.00,0.00,0.00}{#1}}
\newcommand{\ControlFlowTok}[1]{\textcolor[rgb]{0.13,0.29,0.53}{\textbf{#1}}}
\newcommand{\DataTypeTok}[1]{\textcolor[rgb]{0.13,0.29,0.53}{#1}}
\newcommand{\DecValTok}[1]{\textcolor[rgb]{0.00,0.00,0.81}{#1}}
\newcommand{\DocumentationTok}[1]{\textcolor[rgb]{0.56,0.35,0.01}{\textbf{\textit{#1}}}}
\newcommand{\ErrorTok}[1]{\textcolor[rgb]{0.64,0.00,0.00}{\textbf{#1}}}
\newcommand{\ExtensionTok}[1]{#1}
\newcommand{\FloatTok}[1]{\textcolor[rgb]{0.00,0.00,0.81}{#1}}
\newcommand{\FunctionTok}[1]{\textcolor[rgb]{0.00,0.00,0.00}{#1}}
\newcommand{\ImportTok}[1]{#1}
\newcommand{\InformationTok}[1]{\textcolor[rgb]{0.56,0.35,0.01}{\textbf{\textit{#1}}}}
\newcommand{\KeywordTok}[1]{\textcolor[rgb]{0.13,0.29,0.53}{\textbf{#1}}}
\newcommand{\NormalTok}[1]{#1}
\newcommand{\OperatorTok}[1]{\textcolor[rgb]{0.81,0.36,0.00}{\textbf{#1}}}
\newcommand{\OtherTok}[1]{\textcolor[rgb]{0.56,0.35,0.01}{#1}}
\newcommand{\PreprocessorTok}[1]{\textcolor[rgb]{0.56,0.35,0.01}{\textit{#1}}}
\newcommand{\RegionMarkerTok}[1]{#1}
\newcommand{\SpecialCharTok}[1]{\textcolor[rgb]{0.00,0.00,0.00}{#1}}
\newcommand{\SpecialStringTok}[1]{\textcolor[rgb]{0.31,0.60,0.02}{#1}}
\newcommand{\StringTok}[1]{\textcolor[rgb]{0.31,0.60,0.02}{#1}}
\newcommand{\VariableTok}[1]{\textcolor[rgb]{0.00,0.00,0.00}{#1}}
\newcommand{\VerbatimStringTok}[1]{\textcolor[rgb]{0.31,0.60,0.02}{#1}}
\newcommand{\WarningTok}[1]{\textcolor[rgb]{0.56,0.35,0.01}{\textbf{\textit{#1}}}}
\usepackage{longtable,booktabs}
\usepackage{graphicx,grffile}
\makeatletter
\def\maxwidth{\ifdim\Gin@nat@width>\linewidth\linewidth\else\Gin@nat@width\fi}
\def\maxheight{\ifdim\Gin@nat@height>\textheight\textheight\else\Gin@nat@height\fi}
\makeatother
% Scale images if necessary, so that they will not overflow the page
% margins by default, and it is still possible to overwrite the defaults
% using explicit options in \includegraphics[width, height, ...]{}
\setkeys{Gin}{width=\maxwidth,height=\maxheight,keepaspectratio}
\IfFileExists{parskip.sty}{%
\usepackage{parskip}
}{% else
\setlength{\parindent}{0pt}
\setlength{\parskip}{6pt plus 2pt minus 1pt}
}
\setlength{\emergencystretch}{3em}  % prevent overfull lines
\providecommand{\tightlist}{%
  \setlength{\itemsep}{0pt}\setlength{\parskip}{0pt}}
\setcounter{secnumdepth}{0}
% Redefines (sub)paragraphs to behave more like sections
\ifx\paragraph\undefined\else
\let\oldparagraph\paragraph
\renewcommand{\paragraph}[1]{\oldparagraph{#1}\mbox{}}
\fi
\ifx\subparagraph\undefined\else
\let\oldsubparagraph\subparagraph
\renewcommand{\subparagraph}[1]{\oldsubparagraph{#1}\mbox{}}
\fi

%%% Use protect on footnotes to avoid problems with footnotes in titles
\let\rmarkdownfootnote\footnote%
\def\footnote{\protect\rmarkdownfootnote}

%%% Change title format to be more compact
\usepackage{titling}

% Create subtitle command for use in maketitle
\providecommand{\subtitle}[1]{
  \posttitle{
    \begin{center}\large#1\end{center}
    }
}

\setlength{\droptitle}{-2em}

  \title{Homework 10: Multiple Linear Regression}
    \pretitle{\vspace{\droptitle}\centering\huge}
  \posttitle{\par}
    \author{}
    \preauthor{}\postauthor{}
    \date{}
    \predate{}\postdate{}
  

\begin{document}
\maketitle

\begin{center}\rule{0.5\linewidth}{\linethickness}\end{center}

\hypertarget{name-your-name}{%
\section{NAME: Your Name}\label{name-your-name}}

\hypertarget{netid-your-netid}{%
\section{NETID: Your NetID}\label{netid-your-netid}}

\hypertarget{due-date-december-11-2019-by-1159pm}{%
\subsection{\texorpdfstring{\textbf{DUE DATE: December 11, 2019 by
11:59pm
}}{DUE DATE: December 11, 2019 by 11:59pm }}\label{due-date-december-11-2019-by-1159pm}}

\hypertarget{instructions}{%
\subsection{Instructions}\label{instructions}}

For this homework:

\begin{enumerate}
\def\labelenumi{\arabic{enumi}.}
\item
  All calculations must be done within your document in code chunks.
  Provide all intermediate steps.
\item
  DO NOT JUST INCLUDE A CALCULATION: Incude any formulas you are using
  for a calculation. You can put these immediately before the code chunk
  where you actually do the calculation.
\end{enumerate}

\hypertarget{hollywood-movies-2011-dataset}{%
\subsection{Hollywood Movies 2011
Dataset}\label{hollywood-movies-2011-dataset}}

This dataset includes information for 118 movies released in 2011. Here
is a brief description of each of the variables included in this
dataset.

\begin{longtable}[]{@{}ll@{}}
\toprule
\begin{minipage}[b]{0.25\columnwidth}\raggedright
Variable\strut
\end{minipage} & \begin{minipage}[b]{0.61\columnwidth}\raggedright
Description\strut
\end{minipage}\tabularnewline
\midrule
\endhead
\begin{minipage}[t]{0.25\columnwidth}\raggedright
\texttt{WorldGross}\strut
\end{minipage} & \begin{minipage}[t]{0.61\columnwidth}\raggedright
Gross income for all viewers (in millions)\strut
\end{minipage}\tabularnewline
\begin{minipage}[t]{0.25\columnwidth}\raggedright
\texttt{AudienceScore}\strut
\end{minipage} & \begin{minipage}[t]{0.61\columnwidth}\raggedright
Audience Rating\strut
\end{minipage}\tabularnewline
\begin{minipage}[t]{0.25\columnwidth}\raggedright
\texttt{BOAveOpenWeek}\strut
\end{minipage} & \begin{minipage}[t]{0.61\columnwidth}\raggedright
Average box office income per theater in the opening week\strut
\end{minipage}\tabularnewline
\begin{minipage}[t]{0.25\columnwidth}\raggedright
\texttt{Budget}\strut
\end{minipage} & \begin{minipage}[t]{0.61\columnwidth}\raggedright
Production Budget (in millions)\strut
\end{minipage}\tabularnewline
\begin{minipage}[t]{0.25\columnwidth}\raggedright
\texttt{Fantasy}\strut
\end{minipage} & \begin{minipage}[t]{0.61\columnwidth}\raggedright
TRUE if the movie genre is Fantasy; FALSE if the movie genre is not
Fantasy\strut
\end{minipage}\tabularnewline
\bottomrule
\end{longtable}

\hypertarget{problem-1}{%
\subsubsection{Problem 1}\label{problem-1}}

Here we will explore a MLR with response equal to \texttt{WorldGross}
and the following predictors: \texttt{AudienceScore},
\texttt{BOAveOpenWeek}, \texttt{Fantasy}, and \texttt{Budget}.

\begin{enumerate}
\def\labelenumi{\alph{enumi})}
\tightlist
\item
  Read the data into this homework document and list the variable names.
\end{enumerate}

\begin{Shaded}
\begin{Highlighting}[]
\NormalTok{hollywood =}\StringTok{ }\KeywordTok{read.csv}\NormalTok{(}\StringTok{"https://raw.githubusercontent.com/mdarfler/BTRY_6010/master/Homework/HW%2010/Hollywood(3).csv"}\NormalTok{)}

\KeywordTok{names}\NormalTok{(hollywood)}
\end{Highlighting}
\end{Shaded}

\begin{verbatim}
## [1] "AudienceScore" "BOAveOpenWeek" "WorldGross"    "Budget"       
## [5] "Fantasy"
\end{verbatim}

\begin{enumerate}
\def\labelenumi{\alph{enumi})}
\setcounter{enumi}{1}
\tightlist
\item
  Fit a linear model with \texttt{WorldGross} as the response and
  \texttt{AudienceScore}, \texttt{BOAveOpenWeek}, \texttt{Fantasy}, and
  \texttt{Budget} as predictors. Also, include a summary of this model.
\end{enumerate}

\begin{Shaded}
\begin{Highlighting}[]
\NormalTok{hollywood.fit <-}\StringTok{ }\KeywordTok{lm}\NormalTok{(WorldGross }\OperatorTok{~}\StringTok{ }\NormalTok{AudienceScore }\OperatorTok{+}\StringTok{ }\NormalTok{BOAveOpenWeek }\OperatorTok{+}\StringTok{ }\NormalTok{Fantasy }\OperatorTok{+}\StringTok{ }\NormalTok{Budget, }\DataTypeTok{data =}\NormalTok{ hollywood)}

\KeywordTok{summary}\NormalTok{(hollywood.fit)}
\end{Highlighting}
\end{Shaded}

\begin{verbatim}
## 
## Call:
## lm(formula = WorldGross ~ AudienceScore + BOAveOpenWeek + Fantasy + 
##     Budget, data = hollywood)
## 
## Residuals:
##     Min      1Q  Median      3Q     Max 
## -337.46  -53.57   -6.59   48.22  533.89 
## 
## Coefficients:
##                 Estimate Std. Error t value Pr(>|t|)    
## (Intercept)   -1.263e+02  4.525e+01  -2.791  0.00616 ** 
## AudienceScore  1.679e+00  7.129e-01   2.356  0.02020 *  
## BOAveOpenWeek  3.546e-03  1.220e-03   2.907  0.00439 ** 
## FantasyTRUE    8.306e+02  1.319e+02   6.296 6.01e-09 ***
## Budget         2.658e+00  2.432e-01  10.929  < 2e-16 ***
## ---
## Signif. codes:  0 '***' 0.001 '**' 0.01 '*' 0.05 '.' 0.1 ' ' 1
## 
## Residual standard error: 125.4 on 113 degrees of freedom
## Multiple R-squared:  0.6924, Adjusted R-squared:  0.6815 
## F-statistic: 63.59 on 4 and 113 DF,  p-value: < 2.2e-16
\end{verbatim}

\begin{enumerate}
\def\labelenumi{\alph{enumi})}
\setcounter{enumi}{2}
\tightlist
\item
  State the expression for the estimated expected value of
  \texttt{WorldGross} using the model fit in (b).
\end{enumerate}

\begin{quote}
-126.3216251
\end{quote}

\begin{enumerate}
\def\labelenumi{\alph{enumi})}
\setcounter{enumi}{3}
\item
  What values can the covariate \texttt{FantasyTRUE} take on, and what
  is the meaning of each possible value?
\item
  Estimate the expected gross income for a non-fantasy movie that has an
  audience score equal to 90, a budget of 50 million dollars, and that
  has an opening week box office average of \$10,000.
\item
  Use the \texttt{confint()} function to create a 95\% confidence
  interval for the partial slope of \texttt{Budget}. Interpret it in the
  context of this study.
\end{enumerate}

\hypertarget{problem-2}{%
\subsubsection{Problem 2}\label{problem-2}}

Here we will check the assumptions of the MLR fit in Problem 1.

\begin{enumerate}
\def\labelenumi{\alph{enumi})}
\item
  Does it seem reasonable to assume these observations are independent?
\item
  Create a scatterplot of the residuals (on the y-axis) vesus the fitted
  values (on the x-axis). Does the equal variance assumption seem
  reasonable?
\item
  Create a Q-Q plot of the residuals. Does the normality assumption seem
  reasonable?
\item
  Try replacing \texttt{WorldGross} by \texttt{sqrt(WorldGross)} in the
  \texttt{lm} formula. Repeat part (c) and comment.
\item
  Try plotting the residuals of the model in 2d versus the row number of
  the movie (that is, use \texttt{plot} with first argument
  \texttt{1:nrow(Hollywood)} and second argument the residuals). What do
  you observe? Explain what this indicates (and you might want to change
  your answer to 2a accordingly).
\end{enumerate}


\end{document}
