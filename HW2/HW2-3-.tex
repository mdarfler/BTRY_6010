\documentclass[]{article}
\usepackage{lmodern}
\usepackage{amssymb,amsmath}
\usepackage{ifxetex,ifluatex}
\usepackage{fixltx2e} % provides \textsubscript
\ifnum 0\ifxetex 1\fi\ifluatex 1\fi=0 % if pdftex
  \usepackage[T1]{fontenc}
  \usepackage[utf8]{inputenc}
\else % if luatex or xelatex
  \ifxetex
    \usepackage{mathspec}
  \else
    \usepackage{fontspec}
  \fi
  \defaultfontfeatures{Ligatures=TeX,Scale=MatchLowercase}
\fi
% use upquote if available, for straight quotes in verbatim environments
\IfFileExists{upquote.sty}{\usepackage{upquote}}{}
% use microtype if available
\IfFileExists{microtype.sty}{%
\usepackage{microtype}
\UseMicrotypeSet[protrusion]{basicmath} % disable protrusion for tt fonts
}{}
\usepackage[margin=1in]{geometry}
\usepackage{hyperref}
\hypersetup{unicode=true,
            pdftitle={DarflerM-HW2},
            pdfborder={0 0 0},
            breaklinks=true}
\urlstyle{same}  % don't use monospace font for urls
\usepackage{color}
\usepackage{fancyvrb}
\newcommand{\VerbBar}{|}
\newcommand{\VERB}{\Verb[commandchars=\\\{\}]}
\DefineVerbatimEnvironment{Highlighting}{Verbatim}{commandchars=\\\{\}}
% Add ',fontsize=\small' for more characters per line
\usepackage{framed}
\definecolor{shadecolor}{RGB}{248,248,248}
\newenvironment{Shaded}{\begin{snugshade}}{\end{snugshade}}
\newcommand{\AlertTok}[1]{\textcolor[rgb]{0.94,0.16,0.16}{#1}}
\newcommand{\AnnotationTok}[1]{\textcolor[rgb]{0.56,0.35,0.01}{\textbf{\textit{#1}}}}
\newcommand{\AttributeTok}[1]{\textcolor[rgb]{0.77,0.63,0.00}{#1}}
\newcommand{\BaseNTok}[1]{\textcolor[rgb]{0.00,0.00,0.81}{#1}}
\newcommand{\BuiltInTok}[1]{#1}
\newcommand{\CharTok}[1]{\textcolor[rgb]{0.31,0.60,0.02}{#1}}
\newcommand{\CommentTok}[1]{\textcolor[rgb]{0.56,0.35,0.01}{\textit{#1}}}
\newcommand{\CommentVarTok}[1]{\textcolor[rgb]{0.56,0.35,0.01}{\textbf{\textit{#1}}}}
\newcommand{\ConstantTok}[1]{\textcolor[rgb]{0.00,0.00,0.00}{#1}}
\newcommand{\ControlFlowTok}[1]{\textcolor[rgb]{0.13,0.29,0.53}{\textbf{#1}}}
\newcommand{\DataTypeTok}[1]{\textcolor[rgb]{0.13,0.29,0.53}{#1}}
\newcommand{\DecValTok}[1]{\textcolor[rgb]{0.00,0.00,0.81}{#1}}
\newcommand{\DocumentationTok}[1]{\textcolor[rgb]{0.56,0.35,0.01}{\textbf{\textit{#1}}}}
\newcommand{\ErrorTok}[1]{\textcolor[rgb]{0.64,0.00,0.00}{\textbf{#1}}}
\newcommand{\ExtensionTok}[1]{#1}
\newcommand{\FloatTok}[1]{\textcolor[rgb]{0.00,0.00,0.81}{#1}}
\newcommand{\FunctionTok}[1]{\textcolor[rgb]{0.00,0.00,0.00}{#1}}
\newcommand{\ImportTok}[1]{#1}
\newcommand{\InformationTok}[1]{\textcolor[rgb]{0.56,0.35,0.01}{\textbf{\textit{#1}}}}
\newcommand{\KeywordTok}[1]{\textcolor[rgb]{0.13,0.29,0.53}{\textbf{#1}}}
\newcommand{\NormalTok}[1]{#1}
\newcommand{\OperatorTok}[1]{\textcolor[rgb]{0.81,0.36,0.00}{\textbf{#1}}}
\newcommand{\OtherTok}[1]{\textcolor[rgb]{0.56,0.35,0.01}{#1}}
\newcommand{\PreprocessorTok}[1]{\textcolor[rgb]{0.56,0.35,0.01}{\textit{#1}}}
\newcommand{\RegionMarkerTok}[1]{#1}
\newcommand{\SpecialCharTok}[1]{\textcolor[rgb]{0.00,0.00,0.00}{#1}}
\newcommand{\SpecialStringTok}[1]{\textcolor[rgb]{0.31,0.60,0.02}{#1}}
\newcommand{\StringTok}[1]{\textcolor[rgb]{0.31,0.60,0.02}{#1}}
\newcommand{\VariableTok}[1]{\textcolor[rgb]{0.00,0.00,0.00}{#1}}
\newcommand{\VerbatimStringTok}[1]{\textcolor[rgb]{0.31,0.60,0.02}{#1}}
\newcommand{\WarningTok}[1]{\textcolor[rgb]{0.56,0.35,0.01}{\textbf{\textit{#1}}}}
\usepackage{graphicx,grffile}
\makeatletter
\def\maxwidth{\ifdim\Gin@nat@width>\linewidth\linewidth\else\Gin@nat@width\fi}
\def\maxheight{\ifdim\Gin@nat@height>\textheight\textheight\else\Gin@nat@height\fi}
\makeatother
% Scale images if necessary, so that they will not overflow the page
% margins by default, and it is still possible to overwrite the defaults
% using explicit options in \includegraphics[width, height, ...]{}
\setkeys{Gin}{width=\maxwidth,height=\maxheight,keepaspectratio}
\IfFileExists{parskip.sty}{%
\usepackage{parskip}
}{% else
\setlength{\parindent}{0pt}
\setlength{\parskip}{6pt plus 2pt minus 1pt}
}
\setlength{\emergencystretch}{3em}  % prevent overfull lines
\providecommand{\tightlist}{%
  \setlength{\itemsep}{0pt}\setlength{\parskip}{0pt}}
\setcounter{secnumdepth}{0}
% Redefines (sub)paragraphs to behave more like sections
\ifx\paragraph\undefined\else
\let\oldparagraph\paragraph
\renewcommand{\paragraph}[1]{\oldparagraph{#1}\mbox{}}
\fi
\ifx\subparagraph\undefined\else
\let\oldsubparagraph\subparagraph
\renewcommand{\subparagraph}[1]{\oldsubparagraph{#1}\mbox{}}
\fi

%%% Use protect on footnotes to avoid problems with footnotes in titles
\let\rmarkdownfootnote\footnote%
\def\footnote{\protect\rmarkdownfootnote}

%%% Change title format to be more compact
\usepackage{titling}

% Create subtitle command for use in maketitle
\providecommand{\subtitle}[1]{
  \posttitle{
    \begin{center}\large#1\end{center}
    }
}

\setlength{\droptitle}{-2em}

  \title{DarflerM-HW2}
    \pretitle{\vspace{\droptitle}\centering\huge}
  \posttitle{\par}
    \author{}
    \preauthor{}\postauthor{}
    \date{}
    \predate{}\postdate{}
  

\begin{document}
\maketitle

\begin{center}\rule{0.5\linewidth}{\linethickness}\end{center}

\hypertarget{name-michael-darfler}{%
\section{NAME: Michael Darfler}\label{name-michael-darfler}}

\hypertarget{netid-mbd25}{%
\section{NETID: mbd25}\label{netid-mbd25}}

\textbf{DUE DATE: September 14, 2019 by 11:59pm }

\begin{center}\rule{0.5\linewidth}{\linethickness}\end{center}

\hypertarget{homework-2-instructions}{%
\subsection{\texorpdfstring{\textbf{Homework 2 Instructions
}}{Homework 2 Instructions }}\label{homework-2-instructions}}

\begin{enumerate}
\def\labelenumi{\arabic{enumi}.}
\item
  In this homework we will explore the \texttt{StudentSurvey} data
  described below. For each problem:

  \begin{enumerate}
  \def\labelenumii{\alph{enumii})}
  \item
    Answer all questions
  \item
    Insert code chunks directly under any problems that require you to
    use R and type in code as needed. In particular, make sure code
    chunks are included for any requested plots.
  \item
    Answer any questions related to the problem in the .Rmd document
    directly under the question
  \item
    Note: Occasionally when you insert a code chunk it may not go where
    you intend it to. If this happens, you can cut and paste it into the
    correct spot. Make sure the code chunk is aligned to the left margin
    of this document. Often it may be easier to just store a code chunk
    on your clip board and paste it in when you need one.
  \item
    You may need to knit your document occasionally to answer questions
    related to R output.
  \end{enumerate}
\item
  Submit two documents: a R Markdown file and a pdf. These files should
  be named ``\emph{LastF}-HW2.Rmd'' and ``\emph{LastF}-HW2.pdf''.
\end{enumerate}

\#\#SurveyData

An in-class survey was given to 362 introductory statistics students
over several years. The StudentSurvey data contains 17 variables
recorded for each student. They are as follows:

\emph{Year}: Year in school: FirstYear, Sophomore, Junior, or Senior

\emph{Gender}: M or F

\emph{Smoke}: ``No'' or ``Yes''

\emph{Award}: Preferred award: Academy, Nobel or Olympic

\emph{HigherSAT}: Which SAT score was higher: Math or Verbal

\emph{Exercise}: Hours of exercise per week

\emph{TV}: Hours of TV viewing per week

\emph{Height}: Height in inches

\emph{Weight}: Weight in pounds

\emph{Siblings}: Number of Siblings

\emph{BirthOrder}: Birth order: 1=oldest, 2=second oldest, etc.

\emph{VerbalSAT}: Verbal SAT score

\emph{MathSAT}: Math SAT score

\emph{SAT}: Combined Verbal + Math SAT

\emph{GPA}: College GPA

\emph{Pulse}: Pulse Rate (beats per minute)

\emph{Piercings}: Number of body piercings

The \textbf{StudentSurvey} data can be downloaded from the folder for
homework 2 on Blackboard. Put this data set in your folder for homework
2.

To read these data into your R Console:

\begin{enumerate}
\def\labelenumi{\roman{enumi}.}
\item
  In the menu for RStudio above, select \emph{Tools-\textgreater Import
  Dataset-\textgreater From Text File\ldots{}} (depending on the version
  of RStudio you are using, you may need to select
  \emph{File-\textgreater Import Dataset-\textgreater From
  CSV\ldots{}}).
\item
  Navigate to the correct file in your folder for homework 2.
\item
  Click on the StudentSurvey file and choose \emph{Open}.
\item
  A window will pop up where you can preview the data set and possibly
  choose different options for downloading this data. For this data set,
  the defaults are appropriate. Click once on \emph{Import} to read the
  data into the R Console.
\end{enumerate}

You now should see this data listed in the ``Environment'' window in the
upper right corner of RStudio.

\hypertarget{problem-1}{%
\subsubsection{Problem 1}\label{problem-1}}

To read the data into this R Markdown document, we will use the
\texttt{read.csv()} function in R. Fortunately, this function was just
used in the R Console. {[}Some versions of RStudio may use the function
\texttt{read\_csv()} from the library \texttt{readr}.{]}

\begin{enumerate}
\def\labelenumi{\alph{enumi})}
\tightlist
\item
  Create a code chunk here. Copy the code in the R console below that
  starts with \texttt{StudentSurvey\textless{}-read.csv} (or
  \texttt{StudentSurvey\textless{}-read\_csv}) and paste it into this
  code chunk.
\end{enumerate}

\begin{Shaded}
\begin{Highlighting}[]
\NormalTok{SS <-}\StringTok{ }\KeywordTok{read.csv}\NormalTok{(}\StringTok{"https://raw.githubusercontent.com/mdarfler/BTRY_6010/master/HW2/StudentSurvey(3).txt"}\NormalTok{)}

\KeywordTok{names}\NormalTok{(SS)}
\end{Highlighting}
\end{Shaded}

\begin{verbatim}
##  [1] "Year"       "Gender"     "Smoke"      "Award"      "HigherSAT" 
##  [6] "Exercise"   "TV"         "Height"     "Weight"     "Siblings"  
## [11] "BirthOrder" "VerbalSAT"  "MathSAT"    "SAT"        "GPA"       
## [16] "Pulse"      "Piercings"
\end{verbatim}

\begin{Shaded}
\begin{Highlighting}[]
\KeywordTok{dim}\NormalTok{(SS)}
\end{Highlighting}
\end{Shaded}

\begin{verbatim}
## [1] 362  17
\end{verbatim}

\begin{enumerate}
\def\labelenumi{\alph{enumi})}
\setcounter{enumi}{1}
\item
  In the code chunk above also include code to do the following:

  \begin{enumerate}
  \def\labelenumii{\roman{enumii}.}
  \tightlist
  \item
    list the variable names
  \item
    get the dimension of the data
  \end{enumerate}
\item
  Suppose the population of interest is all college students. What would
  you call the sampling method used for this study? How does this affect
  the interpretation of any analysis performed on these data?
\end{enumerate}

\begin{quote}
I would consider this a Non-probability sample because there are many
units in the population whose probability of selection is 0,
e.g.~students NOT in stats. Furthermore, I would consider this a
convenience study because it's a lot easier to reach out to the
population that you already have access too rather than attempting to
randomly sample all university students*
\end{quote}

\begin{enumerate}
\def\labelenumi{\alph{enumi})}
\setcounter{enumi}{3}
\item
  List the variable types of the following (be as specific as
  possible!):

  \begin{enumerate}
  \def\labelenumii{\arabic{enumii}.}
  \item
    TV - \emph{Numerical, Discrete, Ratio, Integer}
  \item
    Award - \emph{Nominal Categorical Factor w/ 3 Levels}
  \item
    Birth Order - \emph{Ordinal, Categorical, Integer}
  \item
    Pulse - \emph{Numerical, Continuous, Ratio, Integer}
  \item
    GPA - \emph{Numerical, Continuous, Ratio, Float}
  \item
    Piercings - \emph{Numerical, Discrete, Ratio, Integer}
  \end{enumerate}
\end{enumerate}

\hypertarget{problem-2}{%
\subsubsection{Problem 2}\label{problem-2}}

One of the questions asked on the \texttt{StudentSurvey} was, ``Which
award would you prefer to win: an Academy Award, a Nobel Prize, or an
Olympic gold medal?''

\begin{enumerate}
\def\labelenumi{\alph{enumi})}
\tightlist
\item
  Which award was most popular amongst students? Create a table of
  counts for \texttt{Award} with R's \texttt{table()} function.
\end{enumerate}

\begin{Shaded}
\begin{Highlighting}[]
\NormalTok{t =}\StringTok{ }\KeywordTok{table}\NormalTok{(SS}\OperatorTok{$}\NormalTok{Award)}
\KeywordTok{print}\NormalTok{(t)}
\end{Highlighting}
\end{Shaded}

\begin{verbatim}
## 
## Academy   Nobel Olympic 
##      31     149     182
\end{verbatim}

\begin{Shaded}
\begin{Highlighting}[]
\KeywordTok{paste0}\NormalTok{(}\StringTok{"The most popular award is "}\NormalTok{, }\KeywordTok{names}\NormalTok{(}\KeywordTok{which}\NormalTok{(t }\OperatorTok{==}\StringTok{ }\KeywordTok{max}\NormalTok{(t))))}
\end{Highlighting}
\end{Shaded}

\begin{verbatim}
## [1] "The most popular award is Olympic"
\end{verbatim}

\begin{enumerate}
\def\labelenumi{\alph{enumi})}
\setcounter{enumi}{1}
\tightlist
\item
  Was the proportion of students preferring each award different for
  women and men? Explain. Complete the following steps to answer this
  question.

  \begin{enumerate}
  \def\labelenumii{\roman{enumii}.}
  \tightlist
  \item
    Create a relative frequency bar chart for \texttt{Award} by
    \texttt{Gender}. The proportions of the preferred awards for each
    gender should sum to 1. You may get the necessary counts using the
    \texttt{table()} function in R, but it may take more than 1 step. Do
    all calculations necessary in the code chunk.
  \item
    Title this chart ``Award by Gender''
  \item
    The bars for Males and Females should be side by side
  \item
    Include a legend for gender using ``F'' and ``M'' as the labels
  \item
    Make the bars vertical
  \item
    Set \texttt{ylim=c(0,\ 0.6)}

    \begin{enumerate}
    \def\labelenumiii{\roman{enumiii}.}
    \setcounter{enumiii}{6}
    \tightlist
    \item
      Include the option, \texttt{args.legend\ =\ list(x="topleft")}
    \item
      Don't forget to answer the question!
    \end{enumerate}
  \end{enumerate}
\end{enumerate}

\begin{Shaded}
\begin{Highlighting}[]
\KeywordTok{barplot}\NormalTok{(}\KeywordTok{prop.table}\NormalTok{(}\KeywordTok{table}\NormalTok{(SS}\OperatorTok{$}\NormalTok{Gender,SS}\OperatorTok{$}\NormalTok{Award)),}
        \DataTypeTok{main =} \StringTok{"Award by Gender"}\NormalTok{,}
        \DataTypeTok{legend.text =} \KeywordTok{c}\NormalTok{(}\KeywordTok{levels}\NormalTok{(SS}\OperatorTok{$}\NormalTok{Gender)),}
        \DataTypeTok{ylim =} \KeywordTok{c}\NormalTok{(}\DecValTok{0}\NormalTok{,}\FloatTok{0.6}\NormalTok{),}
        \DataTypeTok{args.legend =} \KeywordTok{list}\NormalTok{(}\DataTypeTok{x=}\StringTok{"topleft"}\NormalTok{),}
        
\NormalTok{)}
\end{Highlighting}
\end{Shaded}

\includegraphics{HW2-3-_files/figure-latex/unnamed-chunk-3-1.pdf}

\hypertarget{problem-3}{%
\subsubsection{Problem 3}\label{problem-3}}

Another variable recorded for the \texttt{StudentSurvey} is college GPA.
Here we will look at the relationship between college GPA and
\texttt{Award}.

\begin{enumerate}
\def\labelenumi{\alph{enumi})}
\tightlist
\item
  First, create a probability histogram of \texttt{GPA}, set
  \texttt{breaks=50}.
\end{enumerate}

\begin{Shaded}
\begin{Highlighting}[]
\KeywordTok{hist}\NormalTok{(SS}\OperatorTok{$}\NormalTok{GPA, }\DataTypeTok{breaks =} \DecValTok{50}\NormalTok{)}
\end{Highlighting}
\end{Shaded}

\includegraphics{HW2-3-_files/figure-latex/unnamed-chunk-4-1.pdf}

\begin{verbatim}
  i. How would you describe the distribution of GPA?
  
\end{verbatim}

\begin{quote}
I would describe the distribution of this graph as bimodal*
\end{quote}

\begin{verbatim}
  ii. Based on the histogram alone, estimate the range of the most common GPA values.
  
\end{verbatim}

\begin{quote}
I would say that 50\% of the GPA values fall between 2.9 and 3.4*
\end{quote}

\begin{enumerate}
\def\labelenumi{\alph{enumi})}
\setcounter{enumi}{1}
\tightlist
\item
  Create boxplots for \texttt{GPA} separated by \texttt{Award}.
\end{enumerate}

\begin{Shaded}
\begin{Highlighting}[]
\KeywordTok{boxplot}\NormalTok{(SS}\OperatorTok{$}\NormalTok{GPA}\OperatorTok{~}\NormalTok{SS}\OperatorTok{$}\NormalTok{Award,SS)}
\end{Highlighting}
\end{Shaded}

\includegraphics{HW2-3-_files/figure-latex/unnamed-chunk-5-1.pdf}

\begin{enumerate}
\def\labelenumi{\roman{enumi}.}
\tightlist
\item
  Do there appear to be any differences between the median GPAs of the
  three groups? Support your answer using information from the plots.
\end{enumerate}

\begin{quote}
While the median GPA for those who chose Academy and Nobel prizes appear
very similar, the GPA of those who chose the Olympic Medal appear lower
as indicated by the median line in the boxplot.**
\end{quote}

\begin{enumerate}
\def\labelenumi{\roman{enumi}.}
\setcounter{enumi}{1}
\tightlist
\item
  One group has a student whose GPA is different from the others. Which
  group does he/she belong to? What is the statistical term for this
  observation?
\end{enumerate}

\begin{quote}
The outlier in the dataset belongs to the Nobel group. This is indicated
by the circle well below the lower hinge.**
\end{quote}

\hypertarget{problem-4}{%
\subsubsection{Problem 4}\label{problem-4}}

Yet another variable collected by the student survey was
\texttt{Exercise}. This variable recorded the number of hours each
student exercised per week. Here we will look at the relationship
between \texttt{Exercise} and \texttt{Award}.

\begin{enumerate}
\def\labelenumi{\alph{enumi})}
\tightlist
\item
  Use the function \texttt{hist} to create a histogram of the number of
  hours of exercise per week for Sophomores. Be sure to customize the
  plot so that it is clear what it is conveying (e.g., label the axes to
  convey what is being shown) and perhaps adjust \texttt{breaks}
  manually (recall that \texttt{?hist} will give you information about
  the arguments). How would you describe the distribution of
  \texttt{Exercise} for the Sophomore students?
\end{enumerate}

\begin{Shaded}
\begin{Highlighting}[]
\KeywordTok{hist}\NormalTok{(SS}\OperatorTok{$}\NormalTok{Exercise[}\KeywordTok{which}\NormalTok{(SS}\OperatorTok{$}\NormalTok{Year}\OperatorTok{==}\StringTok{"Sophomore"}\NormalTok{)],}
     \DataTypeTok{breaks=}\DecValTok{20}\NormalTok{,}
     \DataTypeTok{main =} \StringTok{"Histogram of hrs/wk of exercise of Sophomores"}\NormalTok{,}
     \DataTypeTok{xlab =} \StringTok{"hr of exercise/week"}
\NormalTok{     )}
\end{Highlighting}
\end{Shaded}

\includegraphics{HW2-3-_files/figure-latex/unnamed-chunk-6-1.pdf}

\begin{quote}
I would describe the distribution as unimodal with a left skew
\end{quote}

\begin{enumerate}
\def\labelenumi{\alph{enumi})}
\setcounter{enumi}{1}
\tightlist
\item
  Use the \texttt{summary} function to get summary statistics for
  \texttt{Exercise}. What was the range of \texttt{Exercise}? If a
  student exercised more hours per week than half of the sample, what is
  the least amount of exercise he/she was getting per week?
\end{enumerate}

\begin{Shaded}
\begin{Highlighting}[]
\KeywordTok{summary}\NormalTok{(SS}\OperatorTok{$}\NormalTok{Exercise)}
\end{Highlighting}
\end{Shaded}

\begin{verbatim}
##    Min. 1st Qu.  Median    Mean 3rd Qu.    Max.    NA's 
##   0.000   5.000   8.000   9.054  12.000  40.000       1
\end{verbatim}

\begin{quote}
The range of exercise is 0 - 40. A student exercising more than half of
sample would have to be exercising at least 10 hrs/week
\end{quote}

\begin{enumerate}
\def\labelenumi{\alph{enumi})}
\setcounter{enumi}{2}
\tightlist
\item
  Create boxplots of \texttt{Exercise} by \texttt{Award}. What can be
  said about the distribution of \texttt{Exercise} for the students who
  preferred to win an Olympic gold medal in comparison to the
  distribution of \texttt{Exercise} for those who chose an Academy Award
  or a Nobel Prize?
\end{enumerate}

\begin{Shaded}
\begin{Highlighting}[]
\KeywordTok{boxplot}\NormalTok{(SS}\OperatorTok{$}\NormalTok{Exercise}\OperatorTok{~}\NormalTok{SS}\OperatorTok{$}\NormalTok{Award,SS)}
\end{Highlighting}
\end{Shaded}

\includegraphics{HW2-3-_files/figure-latex/unnamed-chunk-8-1.pdf}

\begin{quote}
Those who aspire to an olympic medal, on average exercise more than
those who aspire to a nobel prize or an academy award.
\end{quote}

\hypertarget{problem-5}{%
\subsubsection{Problem 5}\label{problem-5}}

Is there a relationship between the number of hours of TV you watch and
your GPA?

\begin{enumerate}
\def\labelenumi{\alph{enumi})}
\tightlist
\item
  Create a scatterplot of \texttt{GPA} by \texttt{TV} using the code
  below.
\end{enumerate}

\begin{Shaded}
\begin{Highlighting}[]
\KeywordTok{plot}\NormalTok{(SS}\OperatorTok{$}\NormalTok{TV, SS}\OperatorTok{$}\NormalTok{GPA)}
\end{Highlighting}
\end{Shaded}

\includegraphics{HW2-3-_files/figure-latex/unnamed-chunk-9-1.pdf}

\begin{enumerate}
\def\labelenumi{\alph{enumi})}
\setcounter{enumi}{1}
\tightlist
\item
  Does there appear to be a relationship between the number of hours
  watching TV and college GPA?
\end{enumerate}

\begin{quote}
There seems to be a compression in the variation of scores as tv
watching increases but the mean appears to remain the same.
\end{quote}

\begin{enumerate}
\def\labelenumi{\alph{enumi})}
\setcounter{enumi}{2}
\tightlist
\item
  Let's take a look at this relationship again after excluding students
  who watches 40 hours of TV a week. Do this in two lines. First, create
  a new data frame (using the \texttt{subset} function). Second, use
  \texttt{plot}.
\end{enumerate}

\begin{Shaded}
\begin{Highlighting}[]
\CommentTok{# First line: create new data frame that keeps only students with TV under 40  }
\NormalTok{ss.subset =}\StringTok{ }\KeywordTok{subset}\NormalTok{(SS,SS}\OperatorTok{$}\NormalTok{TV}\OperatorTok{<}\DecValTok{40}\NormalTok{)}
\CommentTok{# Second line: call plot function with this newly created data frame}
\KeywordTok{plot}\NormalTok{(ss.subset}\OperatorTok{$}\NormalTok{TV,ss.subset}\OperatorTok{$}\NormalTok{GPA)}
\end{Highlighting}
\end{Shaded}

\includegraphics{HW2-3-_files/figure-latex/unnamed-chunk-10-1.pdf}

\begin{enumerate}
\def\labelenumi{\alph{enumi})}
\setcounter{enumi}{3}
\tightlist
\item
  We can look at the difference in this relationship between males and
  females by coloring the female observation red. Is this relationship
  any different for females compared to males? You will do this in three
  lines of code. In the first line, simply repeat the previous call to
  \texttt{plot} that you wrote in part c; in the second line, create a
  data frame called \texttt{females} that only has the rows
  corresponding to women; the third line is written for you.
\end{enumerate}

\begin{Shaded}
\begin{Highlighting}[]
\CommentTok{# First line: Call to plot function from second line of part c}
\KeywordTok{plot}\NormalTok{(ss.subset}\OperatorTok{$}\NormalTok{TV,ss.subset}\OperatorTok{$}\NormalTok{GPA)}
\CommentTok{# Second line: create data frame called females}
\NormalTok{females =}\StringTok{ }\KeywordTok{subset}\NormalTok{(SS,SS}\OperatorTok{$}\NormalTok{Gender}\OperatorTok{==}\StringTok{"F"}\NormalTok{)}
\KeywordTok{points}\NormalTok{(females}\OperatorTok{$}\NormalTok{TV, females}\OperatorTok{$}\NormalTok{GPA, }\DataTypeTok{col=}\StringTok{'red'}\NormalTok{) }\CommentTok{# this is third line}
\end{Highlighting}
\end{Shaded}

\includegraphics{HW2-3-_files/figure-latex/unnamed-chunk-11-1.pdf}

\begin{quote}
There does not seem to be a difference in the the effect of TV watching
on college GPA between males and females.
\end{quote}

\begin{enumerate}
\def\labelenumi{\alph{enumi})}
\setcounter{enumi}{4}
\tightlist
\item
  What was the effect of the \texttt{points()} function above?
  \textgreater{} \texttt{points()} draws points at specified coordinates
  and the argument \texttt{col\ =\ "red"} colors the red.
\end{enumerate}

\hypertarget{problem-6}{%
\subsubsection{Problem 6}\label{problem-6}}

For this problem, we will examine the variable, \texttt{Piercings}.

\begin{enumerate}
\def\labelenumi{\alph{enumi})}
\tightlist
\item
  In R, output from using the \texttt{class()} function on a variable
  tells you what class R has given that variable.
\end{enumerate}

\begin{Shaded}
\begin{Highlighting}[]
\KeywordTok{class}\NormalTok{(SS}\OperatorTok{$}\NormalTok{Year)}
\end{Highlighting}
\end{Shaded}

\begin{verbatim}
## [1] "factor"
\end{verbatim}

\begin{Shaded}
\begin{Highlighting}[]
\KeywordTok{class}\NormalTok{(SS}\OperatorTok{$}\NormalTok{Piercings)}
\end{Highlighting}
\end{Shaded}

\begin{verbatim}
## [1] "integer"
\end{verbatim}

\begin{Shaded}
\begin{Highlighting}[]
\KeywordTok{class}\NormalTok{(SS}\OperatorTok{$}\NormalTok{GPA)}
\end{Highlighting}
\end{Shaded}

\begin{verbatim}
## [1] "numeric"
\end{verbatim}

\begin{enumerate}
\def\labelenumi{\alph{enumi})}
\setcounter{enumi}{1}
\tightlist
\item
  \texttt{Piercings} is of class ``integer''. For this problem, we will
  consider \texttt{Piercings} as a ``factor'' (categorical variable).
  Run the following to change the class of \texttt{Piercings}.
\end{enumerate}

\begin{Shaded}
\begin{Highlighting}[]
\NormalTok{Piercings <-}\StringTok{ }\KeywordTok{as.factor}\NormalTok{(SS}\OperatorTok{$}\NormalTok{Piercings)}
\end{Highlighting}
\end{Shaded}

\begin{enumerate}
\def\labelenumi{\alph{enumi})}
\setcounter{enumi}{2}
\tightlist
\item
  Create a pie chart of \texttt{Piercings} using the following code. Is
  this a good graphical summary of these data? Explain.
\end{enumerate}

\begin{Shaded}
\begin{Highlighting}[]
\KeywordTok{pie}\NormalTok{(}\KeywordTok{table}\NormalTok{(Piercings))}
\end{Highlighting}
\end{Shaded}

\includegraphics{HW2-3-_files/figure-latex/unnamed-chunk-14-1.pdf}

\begin{quote}
It's a pretty terrible image. It's hard to read, or to make observations
and inferences from.
\end{quote}

\begin{enumerate}
\def\labelenumi{\alph{enumi})}
\setcounter{enumi}{3}
\tightlist
\item
  What might be a better way to graphically describe the distribution of
  \texttt{Piercings}?
\end{enumerate}

\begin{quote}
A barplot might be nice. I would have said a histogram, but since we can
no longer considering the values as integers, but rather as factors, the
histogram would not be appropriate.
\end{quote}

\begin{Shaded}
\begin{Highlighting}[]
\KeywordTok{barplot}\NormalTok{(}\KeywordTok{table}\NormalTok{(SS}\OperatorTok{$}\NormalTok{Piercings))}
\end{Highlighting}
\end{Shaded}

\includegraphics{HW2-3-_files/figure-latex/unnamed-chunk-15-1.pdf}

\begin{enumerate}
\def\labelenumi{\alph{enumi})}
\setcounter{enumi}{4}
\tightlist
\item
  Suppose we want to reduce the number of levels for \texttt{Piercings}
  from 11 to 8. What might be the best way to re-group these data so
  that the pie chart is a better representation of the distribution of
  \texttt{Piercings}?
\end{enumerate}

\begin{quote}
Looking at a histogram of the data it becomes clear that very few people
have 7 or more piercings. Therefore it would make the pie chart easier
to ready if we collapse the last 3 factors resulting in the factors
0,1,2,3,4,5,6,7+.
\end{quote}


\end{document}
